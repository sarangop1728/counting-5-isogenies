\documentclass[12pt]{amsart}

% PREAMBLE
\usepackage{mathscinet}
\usepackage[utf8]{inputenc} % encoding
\usepackage[dvipsnames, table]{xcolor} % colors
\usepackage{color} % colors
\usepackage{tikz} % code figures
\usepackage{tikz-cd} % commutative diagrams
\usetikzlibrary{matrix,arrows,decorations.pathmorphing} % special decorations for tikzmain.tex
\usepackage{amssymb, amsmath, mathtools, amsfonts, amsthm} % ams    
\usepackage{stmaryrd}
\usepackage{thmtools,
  thm-restate} % restate theorems and other options
\declaretheorem[numberwithin=section, name = Theorem]{mainthm}
% \usepackage{pdfpages} % inclusion of external PDF documents in LATEX
\usepackage{graphicx} % optional arguments for \includegraphics
\usepackage{fullpage} % sets margins to be either 1 inch or 1.5 cm
\usepackage{caption} % customize the captions in floating environments
\usepackage{latexsym} % for \lhd and \unlhd (and others)
\usepackage{enumitem} % customize items in lists
\usepackage[toc,page]{appendix} % formatting the titles of appendices
\usepackage{multicol} % typeset text in multiple columns
\usepackage{hyperref}
\definecolor{headercolor}{RGB}{255,255,240}
\definecolor{mylinkcolor}{RGB}{0,0,255}
\definecolor{mycitecolor}{RGB}{169,169,169}
\definecolor{myurlcolor}{RGB}{255,20,147}
% see https://latexcolor.com/ for more options
\definecolor{santicolor}{rgb}{0.0, 0.8, 0.6}
\hypersetup{colorlinks=true,
urlcolor=myurlcolor,
citecolor=mycitecolor,
linkcolor=mylinkcolor,
linktoc=page,
breaklinks=true}
\usepackage{url} 
\urlstyle{same}

\usepackage{cleveref}
% \usepackage{chngcntr}
% \counterwithin{figure}{section} %CHANGES FIGURE COUNTERS TO SECTION
% \counterwithin{table}{section}

\usepackage[numbers]{natbib} % bibliography style
\usepackage{setspace}
\setlength{\bibsep}{0.0pt} % compact bibliography
\usepackage{indentfirst} % to indent the first paragraph of a section
\usepackage{listings} % source-code printing
\usepackage[mathscr]{eucal} % nice caligraphy for sheaves and categories
\usepackage{manfnt} % street sign
\usepackage{longtable} % tables that continue on the next page
\usepackage{colonequals} % correct way to write :=
\usepackage{nicefrac} % typeset in-line fractions in a "nice" way. \nicefrac[\mathcal]{numerator}{denominator}
\usepackage{wrapfig}
% \usepackage{draftwatermark} % watermarks for drafts
% \SetWatermarkText{DRAFT} \SetWatermarkScale{1}
%\usepackage{bbold} % to use \mathbb{1}
\usepackage{tkz-euclide} % euclidean geometry
\usepackage{caption} % figures and subfigures
\usepackage{subcaption} % figures and subfigures
\usepackage{float}
\usepackage{soul} % to strikethrough words: \st{}
\usepackage{pgfplots}
\pgfplotsset{width=10cm,compat=1.9} 

% \usepackage{mathpazo}
% \usepackage{lineno} %numering lines
% \linenumbers

\usepackage{listings}
\definecolor{codegreen}{rgb}{0,0.6,0}
\definecolor{codegray}{rgb}{0.5,0.5,0.5}
\definecolor{codepurple}{rgb}{0.58,0,0.82}
\definecolor{backcolour}{rgb}{0.95,0.95,0.92}

\lstdefinestyle{mystyle}{ backgroundcolor=\color{backcolour},   commentstyle=\color
 {codegreen}, keywordstyle=\color{magenta}, numberstyle=\tiny\color{codegray}, stringstyle=\color
 {codepurple}, basicstyle=\ttfamily\footnotesize, breakatwhitespace=false,         breaklines=true,
 captionpos=b,                    keepspaces=true,                 numbers=left, numbersep=5pt,
 showspaces=false,                showstringspaces=false, showtabs=false, tabsize=2 }

\lstset{style=mystyle, language=[Objective]C}


% THEOREM ENVIRONMENTS.

\newcounter{counter}[section] % global counter / increases with subsection.
\renewcommand{\thecounter}{\thesection.\arabic{counter}}


\newenvironment{warning}{\vspace{.5em} \textdbend \textbf{Warning.}}{
  \vspace*{1em}}
\numberwithin{equation}{section} % numbering according to section
\newtheorem{thmx}{Theorem}
\renewcommand{\thethmx}{\Alph{thmx}} % alphabetic numbering
\newtheorem{corox}{Corollary} \renewcommand{\thecorox}{\Alph{corox}}

% \numberwithin{equation}{subsection} % numbering according to subsection
\newtheorem{theorem}[counter]{Theorem}
\newtheorem{lemma}[counter]{Lemma}
\newtheorem{corollary}[counter]{Corollary}
\newtheorem{conjecture}[counter]{Conjecture}
\newtheorem{proposition}[counter]{Proposition}
\newtheorem{question}[counter]{Question}


\theoremstyle{definition} \newtheorem{definition}[counter]{Definition}
\newtheorem{example}[counter]{Example} % examples
\newtheorem{observations}[counter]{Observations}
\newtheorem*{answer}{Answer} \newtheorem*{fact}{N.B}
\newtheorem{assume}[counter]{Assumption} \newtheorem*{claim}{Claim}
\newtheorem*{List}{\underline{List}} \newtheorem*{apply}{Application}


\theoremstyle{remark} \newtheorem{nonexam}[counter]{Non-example}
\newtheorem*{notation}{Notation}
\newtheorem{remark}[counter]{Remark}
\newtheorem{remarks}[counter]{Remarks}




% COMMAND SHORTCUTS.
  
% Random.
\newcommand{\R}{\mathcal{R}} % region
\newcommand{\RR}{\mathbb{R}} % for Real numbers
\newcommand{\ZZ}{\mathbb{Z}} % for Integers
\newcommand{\QQ}{\mathbb{Q}} % for Rationals
\newcommand{\CC}{\mathbb{C}} % for Complex numbers
\newcommand{\OO}{\mathscr{O}} % for ring of integers
\renewcommand{\AA}{\mathbf{A}} % affine space
\newcommand{\PP}{\mathbf{P}} % projective space
\newcommand{\Qp}{\mathbf{Q}_p} % for p-adics
\newcommand{\Qbar}{\bar{\QQ}} % for algebraic numbers
\newcommand{\XX}{\mathscr{X}} % mathscr X
\newcommand{\XOF}{\XX_0(5)} % stacky X0(5)
\newcommand{\XOFt}{\XX_0^{tw}(5)} %rigidification of X0(5)
\newcommand{\YY}{\mathscr{Y}} % mathscr Y
\newcommand{\FF}{\mathscr{F}} % mathscr Y
\newcommand{\GG}{\mathscr{G}} 
\newcommand{\sW}{\mathscr{W}} 
\newcommand{\sE}{\mathscr{E}} 
\newcommand{\Fp}{\mathbf{F}_p} % field with p elements
\newcommand{\Fq}{\mathbf{F}_q} % field with q elements
\newcommand{\Pone}{\mathbf{P}^1} % for projective line
\newcommand{\Aone}{\mathbf{A}^1} % affine line
\newcommand{\degr}{\operatorname{deg }} % for degree
\newcommand{\md}{\text{ mod }} % mod without parenthesis
\newcommand{\inv}{^{-1}} % inverse
\newcommand{\unit}{^{\times}} % unit groupo functor
\newcommand{\dd}{\,\mathrm{d}} % "d" of differential
\newcommand{\wt}[1]{\widetilde{#1}} % wide tilde
\newcommand{\paren}[1]{\left( #1 \right)} % parenthesis
\newcommand{\brk}[1]{\left\lbrace #1 \right\rbrace} % brackets
\newcommand{\sqbrk}[1]{\left[ #1 \right]} % square brackets
\newcommand{\abs}[1]{\left| #1 \right|} % absolute value
\newcommand{\ceil}[1]{\lceil #1 \rceil} % ceiling funtion
\newcommand{\ZmodnZ}[1]{\ZZ / #1 \ZZ} % Z/nZ
\newcommand{\floor}[1]{\left\lfloor #1 \right\rfloor} % floor function
\newcommand{\frob}{\mathrm{Frob}} % frobenius
\newcommand{\G}[1][\QQ]{\mathrm{Gal}_{#1}} % absolute Galois group
\newcommand{\pabs}[1]{\left| #1 \right|_p} % p-adic absolute value
\newcommand{\ideal}[1]{\langle #1 \rangle} % ideal
\newcommand{\cdef}[1]{\textsf{#1}} % highlight words in definitions
\renewcommand{\Re}{\mathrm{Re}} % real part
\renewcommand{\Im}{\mathrm{Im}} % imaginary part
\newcommand\doublelong[2]{\mathbin{\xymatrix{{}\ar@<3pt>[r]^{#1}
\ar@<-3pt>[r]_{#2}&}}} % command needed for rigidification
\newcommand{\thickslash}{\mathbin{\!\!\pmb{\fatslash}}} % symbol for rigidification




\newcommand{\ol}{\overline}
% Mathmode.
\newcommand{\mbb}[1]{\mathbb{#1}} % mathbb
\newcommand{\mbf}[1]{\mathbf{#1}} % mathbf
\newcommand{\msf}[1]{\mathsf{#1}} % mathsf
\newcommand{\mscr}[1]{\mathscr{#1}} % mathscr
\newcommand{\mst}{\text{ such that }} % "such that" text in mathmode
\newcommand{\mwith}{\text{ with }} % "with" in mathmode
\newcommand{\mand}{\text{ and }} % "and" in mathmode
\newcommand{\mfor}{\text{ for }} % "for" in mathmode
\newcommand{\mforall}{ \text{ for all }} % "for all" in mathmode
\newcommand{\mof}{\text{ of }} % "of" in mathmode
\newcommand{\mif}{\text{ if }} % "if" in mathmode
\newcommand{\mby}{\text{ by }} % "by" in mathmode
\newcommand{\mor}{\text{ or }} % "or" in mathmode
\newcommand{\mto}{\text{ to }} % "to" in mathmode
\newcommand{\motherwise}{\text{ otherwise }} % "otherwise" in mathmode

% Groups.
\newcommand{\GL}{\msf{GL}} % General Linear
\newcommand{\SL}{\msf{SL}} % Special Linear
\newcommand{\GSp}{\msf{GSp}} % General Simplectic
\newcommand{\Sp}{\msf{Sp}} % Simplectic
\newcommand{\UU}{\msf{U}} % Unitary
\newcommand{\Gm}{\mathbb{G}_m} % Multiplicative
\newcommand{\Magma}{\texttt{Magma}}
\newcommand{\cN}{\mathcal{N}}
\newcommand{\cP}{\mathcal{P}}
\newcommand{\stkC}{\mathscr{C}} % Stacky C
\newcommand{\stkP}{\mathscr{P}} % Stacky P




% Remove weird space before "\left".
\let\originalleft\left \let\originalright\right
\renewcommand{\left}{\mathopen{}\mathclose\bgroup\originalleft}
  \renewcommand{\right}{\aftergroup\egroup\originalright}

% COMMENTS IN PDF.
\newcommand{\sapedit}[1]{{\color{cyan} \sf #1}}
\newcommand{\santi}[1]{{\color{cyan} \sf
    $\spadesuit$ Santi: #1}}
\newcommand{\margSAP}[1]{\normalsize{{\color{cyan}\footnote{\color{cyan}#1}}{\marginpar[{\color{cyan}\hfill\tiny\thefootnote$\rightarrow$}]{{\color{cyan}$\leftarrow$\tiny\thefootnote}}}}}
\newcommand{\Santi}[1]{\margSAP{(Santiago) #1}}

% COMMENTS IN PDF.
\newcommand{\chanedit}[1]{{\color{teal} \sf #1}}
\newcommand{\chan}[1]{{\color{teal} \sf
    $\clubsuit\clubsuit\clubsuit$ Changho: [#1]}}
\newcommand{\margCHAN}[1]{\normalsize{{\color{teal}\footnote{\color{teal}#1}}{\marginpar[{\color{teal}\hfill\tiny\thefootnote$\rightarrow$}]{{\color{teal}$\leftarrow$\tiny\thefootnote}}}}}
\newcommand{\Changho}[1]{\margCHAN{(Changho) #1}}

\definecolor{orchid}{rgb}{0.85, 0.44, 0.84}

% COMMENTS IN PDF.
\newcommand{\oanaedit}[1]{{\color{orchid} \sf #1}}
\newcommand{\oana}[1]{{\color{orchid} \sf
    $\clubsuit\clubsuit\clubsuit$ Oana: [#1]}}
\newcommand{\margOANA}[1]{\normalsize{{\color{orchid}\footnote{\color{orchid}#1}}{\marginpar[{\color{orchid}\hfill\tiny\thefootnote$\rightarrow$}]{{\color{orchid}$\leftarrow$\tiny\thefootnote}}}}}
\newcommand{\Oana}[1]{\margOANA{(Oana) #1}}

% COMMENTS IN PDF. (Sun Woo)
\newcommand{\sparkedit}[1]{{\color{olive} \sf #1}}
\newcommand{\spark}[1]{{\color{olive} \sf
    $\clubsuit\clubsuit\clubsuit$ Sun Woo: [#1]}}
\newcommand{\margSPARK}[1]{\normalsize{{\color{olive}\footnote{\color{olive}#1}}{\marginpar[{\color{olive}\hfill\tiny\thefootnote$\rightarrow$}]{{\color{olive}$\leftarrow$\tiny\thefootnote}}}}}
\newcommand{\Sunwoo}[1]{\margSPARK{(Sun Woo) #1}}

% OPERATORS
\DeclareMathOperator{\height}{Ht} % height
\DeclareMathOperator{\len}{length} % length
\DeclareMathOperator{\nrm}{N} % norm
\DeclareMathOperator{\Nm}{Nm} % norm
\DeclareMathOperator{\disc}{disc} % discriminant
\DeclareMathOperator{\ord}{ord} % order
\DeclareMathOperator{\car}{char} % characteristic
\DeclareMathOperator{\Aut}{Aut} % automorphism
\DeclareMathOperator{\Gal}{Gal} % galois
\DeclareMathOperator{\id}{id} % identity
\DeclareMathOperator{\Frac}{Frac} % fraction field
\DeclareMathOperator{\splt}{Splt} % split
\DeclareMathOperator{\Br}{Br} % Brauer group
\DeclareMathOperator{\codim}{codim} % codimension
\DeclareMathOperator{\SFab}{Stab} % stabilizer
\DeclareMathOperator{\Tr}{Tr} % trace
\DeclareMathOperator{\Hom}{Hom} % Hom set
\DeclareMathOperator{\Mor}{Mor} % Morphism set
\DeclareMathOperator{\rank}{rk} % rank
\DeclareMathOperator{\img}{Im} % Image
\DeclareMathOperator{\im}{im} % image
\DeclareMathOperator{\Pic}{Pic} % Picard group
\DeclareMathOperator{\Sylow}{Syl} % Sylow
\DeclareMathOperator{\Jac}{Jac} % Jacobian
\DeclareMathOperator{\Spec}{Spec} % Spec
\DeclareMathOperator{\mSpec}{mSpec} % maximal Spec
\DeclareMathOperator{\lcm}{lcm} % least common multiple
\DeclareMathOperator{\sgn}{sgn} % sign
\DeclareMathOperator{\Sym}{Sym} % symmetries
\DeclareMathOperator{\coker}{coker} % cokernel
\DeclareMathOperator{\Div}{Div} % Divisor group
\DeclareMathOperator{\Proj}{Proj} % Proj
\DeclareMathOperator{\supp}{supp} % support
\DeclareMathOperator{\res}{res} % residue
\DeclareMathOperator{\vol}{vol} % volume
\DeclareMathOperator{\Span}{Span} % span
\DeclareMathOperator{\ev}{ev} % evaluation map
\DeclareMathOperator{\dR}{dR} % de Rham
\DeclareMathOperator{\trdeg}{tr.deg} % transcendence degree
\DeclareMathOperator{\Res}{Res} % Residue
\DeclareMathOperator{\Log}{Log} % Logarithm
\DeclareMathOperator{\sheafHom}{\mathscr{H}\text{\kern -1pt
    {om}}} % Sheaf Hom
\DeclareMathOperator{\sProj}{\mathscr{P}\text{\kern -1pt
    {roj}}} % Sheaf Hom
    
\DeclareMathOperator{\mind}{md} % minimality defect
\DeclareMathOperator{\twmind}{tmd} % twist minimality defect
\DeclareMathOperator{\Ht}{Ht} % height
\DeclareMathOperator{\twht}{twHt} % twist height
\DeclareMathOperator{\twerr}{twerr} % twist error




% OPENING
\title{Counting minimal integral points on $x^2+y^2 = z^4$ and 5-isogenies of elliptic curves over the rationals}


\author{Santiago Arango-Piñeros} \address{Department of Mathematics,
  Emory University, Atlanta, GA 30322, USA}
\email{santiago.arango@emory.edu}
\urladdr{\url{https://sarangop1728.github.io/}}

\author{Changho Han}
\address{Pure Mathematics, University of Waterloo, Waterloo, ON N2L 3G1, Canada}
\email{changho.han@uwaterloo.ca}
\urladdr{\url{https://hanchangho.github.io/}}

\author{Oana Padurariu} \address{Department of Mathematics and Statistics,
  Boston University}
\email{oana.adascalitei@yahoo.com}
\urladdr{\url{https://sites.google.com/view/oana-padurariu/home}}
\author{Sun Woo Park} \address{Department of Mathematics, University of Wisconsin-Madison, Madison, WI 53703, USA}
\email{spark483@wisc.edu}
\urladdr{\url{https://sites.google.com/wisc.edu/spark483}}

\begin{document}

\begin{abstract}
  We show that elliptic curves over the rationals equiped with a
  $5$-isogeny can be parametrized by minimal integral solutions to the
  Fermat equation $x^2 + y^2 = z^4$ . We use this parametrization to
  prove that the number of $5$-isogenies with naive height bounded by
  $T$ is $\sim C\cdot T^{1/6}(\log T)^2$ for some explicitly
  computable constant $C>0$. This completes the asymptotic count of
  rational points on the genus zero modular curves $X_0(N)$.
\end{abstract}

\maketitle


\section{Introduction}

\section{Analytic ingredients}\label{sec:analytic-ingredients}
\subsection{Tauberian theorems}\label{sec:tauberian-theorems} Estimates of the partial sums of a Dirichlet series can be obtained from its analytic properties via Tauberian theory. See Theorem 1.1 and Remark 1 in \cite{Murty2024} for a complete proof of the following result.

\begin{theorem}
    \label{thm:tauberian1}
    Let $G(s) = \sum_{n=1}^\infty g(n)n^{-s}$ be a Dirichlet series with
    non-negative coefficients, and suppose that $G(s) = \zeta(s)^kh(s)$ for
    some positive integer $k$, and $h(s)$ a Dirichlet series absolutely
    convergent in $\Re(s) \geq 1$, with $h(1)\neq 0$. Then, as $X \to \infty$,
    \begin{equation*}
        \sum_{n\leq X} g(n) = \dfrac{g_{1}}{(k-1)!}X(\log X)^{k-1} + O(X(\log X)^{k-2}),
    \end{equation*}
    where $g_{1}$ is the residue of $G(s)$ at $s = 1$. \\
    Furthermore, if $G(s)$ has Laurent series expansion
    \begin{equation*}
        G(s+1) = \dfrac{g_{k}}{s^k} + \dfrac{g_{k-1}}{s^{k-1}} + \cdots + g_0 + O_m(|s|^m),
    \end{equation*}
    as $s \to 0^+$, for some $g_i \in \CC$ and $m \in \ZZ_{>0}$, then
    \begin{equation*}
        \sum_{n\leq X} g(n) = X\sum_{j = 1}^k
        \dfrac{\lambda_{k-j}}{(k-j)!}(\log X)^{k-j}
        + O_m\left( \dfrac{X}{(\log X)^m}\right),
    \end{equation*}
    where 
    \begin{equation*}
        \lambda_{k-j} = \sum_{i=0}^{j-1}(-1)^{j-1-i}g_{k-i}.
    \end{equation*}
\end{theorem}

A better control of the function $h(s) = G(s)/\zeta(s)^k$ can yield better
estimates on the error term. The following result is a special case of Théorème
A.1 in \cite{CL&T01}. Recall that a complex function $F(s)$ is said to be of
\cdef{finite order} on a half-plane $\Re(s) > r$ if there exists a constant
$\kappa$ such that
    \begin{equation}
        \label{eq:finite-order}
        |F(s)| = O((1+\Im(s))^\kappa),
    \end{equation}
for $\Re(s) > r$. 

\begin{theorem}
    \label{thm:tauberian2}
    Assume the hypothesis of \Cref{thm:tauberian1}, and assume that the
    follwoing conditions are met:
    \begin{enumerate}[label=(\roman*)]
    \item There exists some $1 > \delta > 0$ such that $h(s)$ admits a
      holomorphic continuation to the half-plane $\Re(s) > 1 - \delta > 0$.
        \item The function $G(s)(s-1)^k/s^{k}$ is of finite order in $\Re(s) > 1-\delta$.
    \end{enumerate}
    Then, for every $\varepsilon >0$, 
    \begin{equation*}
      \sum_{n\leq X} g(n) =
      X\sum_{j = 1}^k \dfrac{\lambda_{k-j}}{(k-j)!}(\log X)^{k-j}
      + O_\varepsilon\left(X^{1-\delta + \varepsilon}\right),
    \end{equation*}
    as $X \to \infty$.
\end{theorem}

\section{Counting twist-minimal integer points on $x^2 + y^2 = z^4$}
\label{sec:counting-twist-min-sols}

\begin{definition}
  \label{def:fermat-triple}
  Since in this article we are concerned with the generalized Fermat equation
  of signature $(2,2,4)$, we will say that a triple of integers $(a,b,c)$ is a
  \cdef{Fermat triple} if it satisfies the equation $x^2 + y^2 = z^4$.
\end{definition}

\begin{definition}[Weighted GCD, twist, and minimality]
  \label{def:weighted-stuff}
  Given a triple of positive integers $r,s,t$, the \cdef{$(r,s,t)$-weighted
  greatest common divisor} of $(a,b,c)\in \ZZ^3$ is the greatest positive
  integer $d$ such that $d^r\mid a$, $d^s \mid b$, and $d^t \mid c$. We say
  that $(a_2,b_2,c_2)$ is an \cdef{$(r,s,t)$-twist} of $(a_1,b_1,c_1)$ if
  there exists an integer $n \neq 0$ such that
  $(a_2,b_2,c_2) = (n^ra_1,n^sb_1,n^tc_1)$. A triple of integers $(a,b,c)$ is
  called \cdef{$(r,s,t)$-minimal} if the $(r,s,t)$-weighted common divisor of
  $(a,b,c)$ is equal to one. Equivalently, if there is no prime $p$ such that
  $a^r \equiv b^s \equiv c^t \equiv 0 \md p$.
\end{definition}

At the moment, we are primarly concerned with the weight $(r,s,t) = (2,2,1)$.
This choice comes from the observation that given an integral solution
$(a,b,c)$ to the generalized Fermat equation $x^2 + y^2 = z^4$, then the twist
$(n^2a,n^2b,nc)$ yields a different one for any integer $n$. By factoring out
the $(2,2,1)$-greatest common divisor of a solution, and twisting by $-1$ when
$c <0$, we can arrive at a choice of a smallest solution.

\begin{definition}[Twist-defect and twist-height]
  An integer triple $(a,b,c)$ is
  \cdef{twist-minimal} if the following conditions are satisfied:
  \begin{itemize}
   \item $a^2 + b^2 = c^4$,
   \item $(a,b,c)$ is $(2,2,1)$-minimal, and
   \item $c>0$.
   \end{itemize}
     The \cdef{twist minimality defect} of a triple $(a,b,c)$ is its
  $(2,2,1)$-greatest common divisor, and it is denoted by $\twmind(a,b,c)$. By
  definition, every integral solution $(a,b,c)$ of $x^2 + y^2 = z^4$ is a
  $(2,2,1)$-twist of the twist-minimal solution $(a/e^2, b/e^2, |c|/e)$, where
  $e = \twmind(a,b,c)$.

  Define the \cdef{twist height} of a Fermat triple $(a,b,c)$ by
  \begin{equation}
    \label{eq:twist-height}
    \twht(a,b,c) \colonequals \dfrac{|c|}{\twmind(a,b,c)}.
  \end{equation}
\end{definition}

The main result of this section is the asymptotic count of twist minimal
triples. For a real parameter $T > 0$, denote by $N_\FF(T)$ the number of twist
minimal triples with twist height less than or equal to $T$. We will prove the
following theorem.

\begin{theorem}
    \label{thm:FF-count}
    There exist explicitly computable constants $f_1, f_2 > 0$ such that for
    every $\epsilon > 0$, we have
    $$N_{\FF}(T) = f_1 T(\log T) + f_2T + O_\epsilon(T^{1/2+\epsilon}),$$
    as $T \to \infty$. Furthermore, the constant $f_1$ is given by 
    \begin{equation*}
      f_1 = \dfrac{\gamma_0\pi^2}{8}\prod_{p \, \equiv\, 3 \md 4}\paren{1-p^{-2}}^2
      \prod_{p \, \equiv\, 1 \md 4}\paren{1 -6p^{-2}+8p^{-3}-3p^{-4}},
      \end{equation*}
      where $\gamma_0$ is the Euler-Mascheroni constant.
\end{theorem}

Before diving into the proof, we record some observations about twist-minimal
triples that will be used in the argument and later on in the article.

\begin{lemma}[Characterization of twist-minimal triples]
  \label{lemma:twist-min-triples}
  Suppose that $(a,b,c)$ is a twist-minimal triple. Then, the
  following are true.
  \begin{enumerate}[label=(\alph*)]
  \item $\gcd(a,b)$ is square free.
  \item If $p \mid \gcd(a,b)$, then $p \mid c$.
  \item If $p \mid c$, then $p \equiv 1 \md 4$.
  \item $c \equiv 1 \md 4$.
  \item $a$ and $b$ have distinct parity.
  \item If $a = 0$, then $b^2 = c^2 = 1$.
  \item If $b = 0$, then $a^2 = c^2 = 1$.
  \end{enumerate}
\end{lemma}
\begin{proof}
  \hfill
  \begin{enumerate}[label=(\alph*)]
  \item If $p^2$ divides both $a$ and $b$, then $p^4 \mid c^4$, which
    implies that $c$ is divisible by $p$. This contradicts the
    $(2,2,1)$-minimality of $(a,b,c)$.
  \item It follows from $a^2 + b^2 = c^4$.
  \item Suppose that $2$ divides $c$, and write $v = v_2(c) > 0$. Then,
    $2^{4v} \mid a^2 + b^2 = (a+ib)(a-ib)$ and we have the equation of ideals
    $\langle a+ib\rangle = \langle 1+i\rangle^{4v}\cdot J = \langle
    2\rangle^{2v}\cdot J$ for some ideal $J \subset \ZZ[i]$. This shows that
    $2^2$ divides $a$ and $b$, implying that $(a,b,c)$ is not $(2,2,1)$-minimal
    at $p=2$, contradiction. If $p \equiv 3 \md 4$ and $p$ divides $c$, then
    $p$ is prime in $\ZZ[i]$ and this implies that $p^{2}$ divides $a + ib$.
    Thus, $p^2$ divides both $a$ and $b$, making it impossible for $(a,b,c)$ to
    be twist-minimal at $p$.
  \item Every prime factor of $c$ is $1 \md 4$, and $c > 0$.
  \item It follows from $a^2 + b^2 \equiv 1 \md 4$.
  \item If $a = 0$, then $a^2 = c^4$. This implies that $|a|$ is a
    square, and if $p \mid a$, then $p^2 \mid a$ and $p \mid c$. The
    $(2,2,1)$-minimality implies that $a$ has no prime divisors.
  \item Same as above.
  \end{enumerate}
\end{proof}

Define the \cdef{height zeta function} corresponding to \Cref{eq:twist-height}
to be the Dirichlet series
\begin{equation*}
  F(s) \colonequals \sum_{c = 1}^\infty \dfrac{f(c)}{c^s},
\end{equation*}
where $f(c)$ is the \cdef{arithmetic height function} that counts the number of
twist-minimal triples $(a',b',c')$ with $\twht(a',b',c') = c' = c$. The
following lemma describes the function $f(c)$ in terms of the prime
factorization of $c$. Recall that an arithmetic function $\phi \colon \ZZ_{>0}
\to \CC$ is \cdef{multiplicative} if $\phi(nm) = \phi(n)\phi(m)$ for every
relatively prime positive integers $n,m$. If moreover $\phi(nm)=\phi(n)\phi(m)$
for every $n,m \in \ZZ_{>0}$, then $\phi$ is called \cdef{completely
multiplicative}.


\begin{lemma}
  \label{lemma:arithmetic-height-fun-f}
    The arithmetic height function $f(c)$ satisfies the following properties.
    \begin{enumerate}
        \item If $p \not\equiv 1 \md 4$, then $f(p^k) = 0$ for every positive integer $k$.
        \item If $p \equiv 1 \md 4$, then $f(p^k) = 16$ for every positive integer $k$.
        \item The arithmetic function $f(c)/4$ is multiplicative, but not
          completely multiplicative.
    \end{enumerate}
\end{lemma}
\begin{proof}
    \hfill
    \begin{enumerate}
        \item Follows from \Cref{lemma:twist-min-triples}.
        \item For such a prime $p$, write $p\ZZ[i] = \pi\cdot\overline{\pi}$
          for the prime ideal factorization in the Gaussian ring. There are
          four ideals $I \subset \ZZ[i]$ that are not divisible by
          $\langle p \rangle^2 = \pi^2\cdot\overline{\pi}^2$ satisfying
          $N(I) \colonequals \#(\ZZ[i]/I) = p^{4k}$; they are:
        \begin{equation*}
          \pi^{4k}, \overline{\pi}^{4k}, \pi^{4k-1}\cdot\overline{\pi},
          \text{ and } \pi\cdot\overline{\pi}^{4k-1}.
        \end{equation*}
        Since there are four units in the principal ideal domain $\ZZ[i]$,
        we see that $f(c)/4$ counts the number of integral ideals $I \subset
        \ZZ[i]$ satisfying:
        \begin{itemize}
          \item $N(I) = c^4$, and
          \item $\langle p \rangle^2$ does not divide $I$ for any rational prime $p$.
        \end{itemize}
        \item This follows directly from the characterization of $f(c)/4$ given
          above, and the multiplicativity of the ideal norm.
    \end{enumerate}
\end{proof}

\begin{proof}[Proof of {\Cref{thm:FF-count}}]
    It follows from the multiplicativity of $f(c)/4$
    (\Cref{lemma:arithmetic-height-fun-f}), and the calculation of prime power
    values of $f$ (\Cref{lemma:twist-min-triples}), that the
    height zeta function $F(s)$ has the following Euler product expansion:
    \begin{align*}
      F(s) = 4\sum_{c=1}^\infty \tfrac14 f(c)c^{-s} =
      & \, 4\prod_{p}\paren{1 + \sum_{k=1}^\infty \tfrac14 f(p^k)p^{-ks} } \\
    = & \, 4\prod_{p \, \equiv\, 1 \md 4}\paren{1 + \sum_{k=1}^\infty 4p^{-ks}}
        = 4\prod_{p \, \equiv\, 1 \md 4}\paren{\dfrac{1+3p^{-s}}{1-p^{-s}}}.
    \end{align*}
    In particular, this calculation shows that $F(s)$ is absolutely convergent
    in the half-plane $\Re(s) > 1$. We can manipulate this Euler product to obtain
    \begin{equation*}
      F(s) = 4 \prod_{p \, \equiv\, 1 \md 4} \paren{1-p^{-s}}^{-4}
      \prod_{p \, \equiv\, 1 \md 4}\paren{1 -6p^{-2s}+8p^{-3s}-3p^{-4s}}.
    \end{equation*}
    At this stage, we introduce the Dedekind zeta function $\zeta_{\QQ(i)}(s)$
    of the Gaussian field $\QQ(i)$. If $\chi_4(n) = (-4 | n)$ is the
    non-trivial Dirichlet character of modulus $4$, we have the well-known
    identity $\zeta_{\QQ(i)}(s) = \zeta(s)L(s,\chi_4)$, where $\zeta(s)$ is
    the Riemann zeta function, and $L(s,\chi_4)$ is the Dirichlet $L$-function
    associated to $\chi_4(n)$.  We deduce the identity
    \begin{equation}
        \label{eq:F=xi.P}
        F(s) = \zeta(s)^2P(s),
    \end{equation}
    where
    \begin{equation*}
      P(s) \colonequals 4L(s,\chi_4)^2\paren{1-2^{-s}}^2
      \prod_{p \, \equiv\, 3 \md 4}\paren{1-p^{-2s}}^2
      \prod_{p \, \equiv\, 1 \md 4}\paren{1 -6p^{-2s}+8p^{-3s}-3p^{-4s}}.
    \end{equation*}
    From \Cref{eq:F=xi.P}, we see that $F(s)$ has a unique pole at $s = 1$
    of order $2$, and admits a meromorphic continuation to the half-plane
    $\Re(s) > 1/2$. Moreover, the residue of $F(s)$ at $s=1$ is
    \begin{align*}
      f_1 &= \Res_{s=1}(F(s)) = \Res_{s=1}(\zeta(s)^2)\cdot P(1) = 2\gamma_0P(1). \\
      P(1) &= 4\cdot(\pi/4)^2\cdot(1/4)\cdot \prod_{p \, \equiv\, 3 \md 4}\paren{1-p^{-2}}^2
      \prod_{p \, \equiv\, 1 \md 4}\paren{1 -6p^{-2}+8p^{-3}-3p^{-4}}.
    \end{align*}


    At this point, we can apply \Cref{thm:tauberian1} and obtain:
    \begin{equation*}
        N_{\FF}(T) = f_1 T(\log T) + f_2T + O\left(\dfrac{T}{\log T}\right).
      \end{equation*}
      To improve this error term, we can use \Cref{thm:tauberian2} with $k = 2$
      and $\delta = 1/2$ to conclude that for any $\epsilon > 0$,
      \begin{equation*}
        N_{\FF}(T) = f_1 T(\log T) + f_2T + O_\epsilon\left(T^{1/2 + \epsilon}\right).
      \end{equation*}
\end{proof}
    

\section{Counting minimal integer points on $x^2 + y^2 = z^4$}
\label{sec:counting-min-sols}

\begin{definition}
  \label{def:minimal-triple}
  We will say that a triple of integers $(a,b,c)$ is a
  \cdef{minimal triple} if the following conditions are satisfied:
  \begin{itemize}
   \item $a^2 + b^2 = c^4$,
   \item $(a,b,c)$ is $(4,4,2)$-minimal, and
   \item $c\neq0$.
  \end{itemize}
  The \cdef{minimality defect} of a triple $(a,b,c)$ is its $(4,4,2)$-greatest
  common divisor, and it is dfenoted by $\mind(a,b,c)$. By definition, every
  integral solution $(a,b,c)$ of $x^2 + y^2 = z^4$ is a $(4,4,2)$-twist of the
  twist-minimal solution $(a/d^4, b/d^4, c/d^2)$, where $d = \mind(a,b,c)$.
\end{definition}

The main result of this section is the asymptotic count of minimal triples. For
a real parameter $T > 0$, denote by $N_\GG(T)$ the number of twist minimal
triples with height less than or equal to $T$. We will prove the following
theorem.

\begin{theorem}
    \label{thm:GG-count}
    There exist explicitly computable constants $g_1, g_2, g_3 > 0$ such that for
    every $\epsilon > 0$, we have
    $$N_{\GG}(T) = g_1 T(\log T)^2 + g_2T(\log T) + g_3T + O_\epsilon(T^{1/2+\epsilon}),$$
    as $T \to \infty$. Furthermore, the constant $g_1$ is given by
    \begin{equation*}
      g_1 = \tfrac{9}{4}(\gamma_0^2+\gamma_1)\prod_{p \, \equiv\, 3 \md 4}\paren{1-p^{-2}}^2
      \prod_{p \, \equiv\, 1 \md 4}\paren{1 -6p^{-2}+8p^{-3}-3p^{-4}}.
      \end{equation*}
      where $\gamma_0$ is the Euler-Mascheroni constant.
\end{theorem}

As before, we aim to understand the analytic properties of the correspondig
\cdef{height zeta function}
\begin{equation*}
    G(s) \colonequals \sum_{n=1}^\infty \frac{g(n)}{n^{s}}, 
\end{equation*}
where $g(n)$ is the \cdef{arithmetic height function} that counts the number of
minimal triples $(a,b,c)$ with $\Ht(a,b,c) = |c| = n$.

Instead of directly analyzing the zeta function $G(s)$, we leverage our
understanding of $F(s)$, in the spirit of \cite[Theorem 5.1.4]{MV22}.

\begin{proof}[Proof of {\Cref{thm:GG-count}}]
    For a positive integer $n$, let 
    \begin{align*}
        \FF_n &\colonequals \brk{(a,b,c) : \text{twist-minimal, with } |c| = c = n}, \\
        \GG_n &\colonequals \brk{(a,b,c) : \text{minimal, with } |c| = n}
    \end{align*}
    be the level sets of points of height $n$, so that $f(n) = \#\FF_n$ and
    $g(n) = \#\GG_n$. For every square-free $d \in \ZZ$, let
    \begin{equation*}
        g^{(d)}(n) \colonequals \#\brk{(a,b,c) \in \ZZ^3 : (d^2a, d^2b, dc) \in \GG_n}.
    \end{equation*}
    It follows from the definitions that
    \begin{equation}
        g^{(d)}(n) = 
        \begin{cases}
            2f(n/|d|), & \mif d \mid n, \\
            0, & \mif d \nmid n.
        \end{cases}
    \end{equation}
    Therefore,
    \begin{align*}
        g(n) = \sum_{\substack{d \in \ZZ \\ \square\text{-free}}} g^{(d)}(n) = 4\sum_{\substack{d > 0 \\ d \mid n}} \mu^2(d)f(n/d) = 4(\mu^2*f)(n).
    \end{align*}
    It follows that
    $G(s) = 4\zeta(s)F(s)/\zeta(2s) = 4\zeta^3(s)P(s)/\zeta(2s)$, and applying
    the Tauberian theorem once more gives the result. In particular, we have
    that
    \begin{align*}
      g_1 &= \tfrac12 \Res_{s=1}(G(s)) \\
          &= 2\Res_{s=1}(\zeta(s)^3)\cdot P(1)\cdot\tfrac{6}{\pi^2}\\
          &= \tfrac{9}{4}(\gamma_0^2+\gamma_1)\prod_{p \, \equiv\, 3 \md 4}\paren{1-p^{-2}}^2
      \prod_{p \, \equiv\, 1 \md 4}\paren{1 -6p^{-2}+8p^{-3}-3p^{-4}}.
    \end{align*}
\end{proof}
    

\section{Elliptic curves and $5$-isogenies}
\label{sec:elliptic-curves-5isog}

We use the notation established in \cite{MV22}. Let $E$ be an elliptic curve
over $\QQ$. Then $E$ admits a short Weierstrass model $E_{A,B}\colon y^2 = x^3
+ Ax + B$ with $A,B \in \QQ$. Furthermore, two such models are isomorphic if
their coefficients satisfy
\begin{equation*}
  (A', B') = (u^4A, u^6B) \text{ for some } u \in \QQ\unit.
\end{equation*}
It follows that we can choose a Weierstrass pair $(A,B) \in \ZZ^2$ to represent
the $\QQ$-isomorphism class of $E$. The \cdef{height} of $(A,B)$ is
$H(A,B) \colonequals \max(4|A|^3, 27|B|^2)$. The \cdef{minimality defect} of
$(A,B)$, denoted $\mind(A,B)$, is the $(4,6)$-weighted common divisor of $A$
and $B$, i.e., largest $d \in \ZZ_{>0}$ such that $d^4 \mid A$ and
$d^6 \mid B$. Every elliptic curve $E$ defined over $\QQ$ with has a unique
short Weierstrass model over, up to $\QQ$-isomorphism, with minimality defect
$1$. We call this the \cdef{minimal model} of $E$, and it is given by
\begin{equation}
    \label{eq:minimal-model}
    y^2 = x^3 + (A/d^4)x + B/d^6, \quad \text{ for } d = \mind(A,B),
\end{equation}
where $E_{A,B}$ is any short Weierstrass model for $E$. We call
\Cref{eq:minimal-model} the \cdef{minimal model} of $E$.




The \cdef{twist minimality defect} of
$(A,B)$, denoted $\twmind(A,B)$, is the $(2,3)$-weighted common divisor of $A$
and $B$, i.e., the largest $e \in \ZZ_{>0}$ such that $e^2 \mid A$ and
$e^3 \mid B$.

For any elliptic curve $E$ over $\QQ$ we choose a short Weierstrass model
$E_{A,B}$ and define the \cdef{height} of (the isomorphism class of) $E$ to be
\begin{equation}
    \label{eq:height}
    \height(E) = \height(A,B) \colonequals H(A,B)/\mind(A,B)^{12}.
\end{equation}
Similarly, the \cdef{twist height} of (the $\Qbar$-isomorphism class of) $E$ is
\begin{equation}
    \label{eq:twist-height}
    \twht(E) = H(A,B)/\twmind(A,B)^{6}.
\end{equation}
For $j = 0, 1728$, we choose twist minimal models as follows:
\begin{itemize}
\item If $j(E) = 0$ ($A =0$), then we take $y^2 = x^3 + 1$ of \cdef{twist
    height} $27$.
\item If $j(E) = 1728$ ($B=0$), then we take $y^2 = x^3 + x$ of \cdef{twist
    height} $4$.
\end{itemize}
Every elliptic curve $E$ defined over $\QQ$ has a unique model over $\QQ$, up
to $\Qbar$-isomorphism, with twist minimality defect $1$ and $B > 0$. We call
this the \cdef{twist minimal model} of $E$, and it is given by
\begin{equation}
    \label{eq:twist-minimal-model}
    y^2 = x^3 + (A/e^2)x + |B|/e^3, \quad \text{ for } e = \twmind(A,B).
\end{equation}

The coarse space of $Y_0(5)$ is an affine open in $\Pone$, so the objects of
interest are parametrized by it's coordinate $t \neq 0, \infty$.
\begin{lemma} \label{lemma:X0(5)-parametrization} The set of isomorphism
  classes of elliptic curves $E$ over $\QQ$ that admit a $5$-isogeny (defined
  over $\QQ$) are precisely those of the form
  $E_5^{(e)}\colon y^2 = x^3 + e^2f(t)x + e^3g(t)$ for some $e \in \ZZ$ square
  free and $t\in \QQ\unit$, where
    \begin{align}
    \label{eq:f}
        f(t) &= -3(t^2 + 10t + 5)(t^2 + 22t + 125), \\
    \label{eq:g}
        g(t) &= 2(t^2 + 4t - 1)(t^2 + 22t + 125)^2.
    \end{align}
\end{lemma}
\begin{proof}
  Elliptic curves over $\QQ$ admitting a $5$-isogeny have $j$-invariant of the
  form (see \cite[Page 1247]{halberstadt21})
    \begin{equation}
        \label{eq:j}
        j = (t^2 + 10t + 5)^3 / t.
    \end{equation}
    From this we get a minimal Weierstrass equation
    $E_5\colon y^2 = x^3 + f(t)x + g(t)$ with $f(t), g(t) \in \ZZ[t]$ as in the
    assertion. A calculation shows that $\disc(E_5) = 12^6t(t^2 + 10t +5)^3$,
    so that the only values of $t\in \QQ$ for which $E_5$ specializes to a
    singular cubic curve is $t=0$. For future reference, let
    \begin{equation}
        \label{eq:h}
        h(t) \colonequals t^2 + 22t + 125.
    \end{equation}
    This polynomial is the greatest common divisor between $f(t)$ and $g(t)$.
    The $5$-division polynomial of the curve $E_2$ has an irreducible quadratic
    factor:
    \begin{equation}
        x^2 -2h(t)x +(t^2 + 22t + 89)h(t),
    \end{equation}
    verifying the claim that the curves $E_5^{(e)}(t)$ admit a five isogeny.
\end{proof}



\section{Minimal triples and $5$-isogenies}
\label{sec:triples-and-5isog}

Given a triple of integers $(a,b,c)$ satisfying the equation $x^2 +
y^2 = z^4$, we define a short Weierstrass equation
\begin{equation}
  \label{eq:Eabc}
  y^2 = x^3 + A(a,b,c)x + B(a,b,c),
\end{equation}
where
\begin{align}
  \label{eq:A}
  A(a,b,c) &\colonequals -6(123a + 114b + 125c^2) \\
  &= -(2\cdot 3^2\cdot 41)a - (2^2\cdot 3^2 \cdot 19)b - (2\cdot 3\cdot 5^3)c^2, \notag\\
  \label{eq:B}
  B(a,b,c) &\colonequals 8c(2502a + 261b + 2500c^2) \\
  &= (2^4\cdot 3^2\cdot 139)ac + (2^3\cdot 3^2\cdot 29)bc + (2^5\cdot 5^4)c^3. \notag
\end{align}
The discriminant of this equation is $\Delta(a,b,c) \colonequals
-16(27A(a,b,c)^3 + 4B(a,b,c)^2)$. When this quantity is not zero,
\Cref{eq:Eabc} defines an elliptic curve $E_{a,b,c}$ over $\QQ$.

\begin{theorem}
  \label{thm:Eabc}
  Suppose $(a,b,c)$ is an integer solution to $x^2 + y^2 = z^4$ that is not
  $(2,2,1)$-equivalent to $(-1,0,1)$ or $(-585,220,25)$. Then,
  \begin{enumerate}[label=(\alph*)]
      \item $\Delta(a,b,c)
  \neq 0$ and $E_{a,b,c}$ is an elliptic curve over $\QQ$
  admitting a $5$-isogeny $\phi_{a,b,c}$.
      \item Two triples are $(2,2,1)$-equivalent by $0 \neq d\in \ZZ$, say
  $(a_2,b_2,c_2) = (d^2a_1,d^2b_1,dc_1)$, if and only if the corresponding
  pairs $(E_1,\phi_1)$ and $(E_2,\phi_2)$ are $\QQ(\sqrt{d})$-twists.
      \item Every pair $(E,\phi) \in Y_0(5)(\QQ)$ is $\QQ$-isomorphic to
  $(E_{a,b,c},\phi_{a,b,c})$ for a unique minimal triple $(a,b,c)$.
      \item The set of quadratic twists of a pair $(E_{a,b,c},\phi_{a,b,c}) \in
        Y_0(5)(\QQ)$ corresponds to the set of $(2,2,1)$-twists of the minimal
        triple $(a,b,c)$.
  \end{enumerate}
\end{theorem}

\section{Counting $5$-isogenies up to $\Qbar$-twist}
\label{sec:counting-5-isogenies-twist}

\begin{lemma}
  \label{lemma:p-mid-c-and-pp-mid-A}
  Let $(a,b,c)$ be a twist-minimal triple. If a prime $p$ satisfies
  \begin{equation*}
    p \mid c,\quad \mand \quad p^2 \mid A(a,b,c),
  \end{equation*}
  then $p=5$.
\end{lemma}
\begin{proof}
  We know from \Cref{lemma:twist-min-triples} that every prime $p$ dividing $c$
  satisfies $p \equiv 1 \md 4$, so it is not possible to have $p = 2,3$
  dividing $c$. Thus, we already know that $p \neq 2,3$.

  Assume that $p \neq 5$. It follows from reducing \Cref{eq:A} modulo $p^2$
  that $123a \equiv -114b \md p^2$. This implies that
  \begin{align*}
    (123c^2)^2 &= (123a)^2 + (123b)^2 \\
               &\equiv (-114b)^2 + (123b)^2 \md p^2 \\
               &\equiv (114^2 + 123^2)b^2 \md p^2 \\
               &\equiv 3^2\cdot 5^5\cdot b^2 \equiv 0 \md p^2 .
  \end{align*}

It follows that $p \mid b$, and therefore $p \mid a$ as well. Let $a'
\colonequals a/p$, $b' \colonequals b/p$ and $c'\colonequals c/p$. We have that
$(a')^2 + (b')^2 = p^2(c')^4$ and $123a' + 114b' + 125p(c')^2 \equiv 0 \md p$.
Once more, it follows that $123a' \equiv -114b' \md p$, and
  \begin{align*}
    (123a')^2 + (123b')^2 &\equiv (114^2 + 123^2)(b')^2 \md p \\
                       &\equiv 3^2\cdot 5^5\cdot (b')^2 \equiv 0 \md p .
  \end{align*}
  It follows that $p \mid b'$, and therefore $p \mid a'$ as well. This
  contradicts the twist-minimality of the triple $(a,b,c)$.

  To see that the case $p=5$ indeed occurs, one can take $(a,b,c) = (-527,
  -336, 25)$. This triple is twist-minimal, since $\gcd(a,b)=1$, and $A(a,b,c)
  = 2^4\cdot 3 \cdot 5^5$.
\end{proof}

\begin{definition}
  Define the \cdef{twist error} of a Fermat triple $(a,b,c)$, denoted by $\twerr(a,b,c)$, to be the twist
  minimality defect of the elliptic curve $E_{a,b,c}$. In other words,
  \begin{equation*}
    \twerr(a,b,c) \colonequals \twmind(A(a,b,c),B(a,b,c)).
  \end{equation*}
\end{definition}

\begin{lemma}
  Let $(a,b,c)$ be a twist-minimal triple. If $p$ is a prime divisor of
  $\twerr(a,b,c)$, then $p \in \brk{2,5}$.
\end{lemma}
\begin{proof}
  By definition, $p \mid \twerr(a,b,c)$ if and only if
  $p^2 \mid A \colonequals A(a,b,c)$ and $p^3 \mid B \colonequals B(a,b,c)$. If we had that $p$ also
  divides $c$, then \Cref{lemma:p-mid-c-and-pp-mid-A} implies that
  $p = 5$. For this reason, we assume that $p$ does not divide $c$.

  We use the equations $A \equiv 0 \md p^2$ and $B \equiv 0 \md p^2$ to find
  congruence relations between $a$ and $b$ modulo $p^2$. We find an integer
  linear combinations of $cA$ and $B$ that allow us to cancel the terms with
  $a$ or $b$ in them, obtaining:
  \begin{align*}
    1112cA + 41B &= -(14000c^3 + 675000bc) = -2^35^3c(14c^2 + 675b) \equiv 0 \md
    p^2, \\
    58ca + 19B &= 337500ac + 336500c^3 = 2^25^3c(673c^2+675a) \equiv 0 \md p^2.
  \end{align*}
  If $p \neq \brk{2,5}$, this implies that $(14c^2 + 675b) \equiv 0 \md
    p^2$ and $(673c^2+675a) \equiv 0 \md p^2$. But in this case, we can reduce
    the equation $(675a)^2 + (675b)^2 = 675^2c^4$ modulo $p^2$ to obtain
    \begin{equation*}
      5^6\cdot 29 = 14^2 + 673^2 \equiv 675^2 = 3^6\cdot 5^4  \md p^2,
    \end{equation*}
    producing a contradiction. We conclude that in this case $p \in \brk{2,5}$,
    and the result follows.
\end{proof}

\santi{Note that $7a + b \equiv 0 \md 25$ is equivalent to $a \equiv 7b \md
  25$.}
\begin{lemma}
  \label{lemma:exceptional-triples}
  Let $(a,b,c)$ be a twist-minimal triples.
  \begin{enumerate}[label=(\alph*)]
  \item \label{item:2-exceptional} $2 \mid \twerr(a,b,c)$ if and only if $2 \mid b$.
  \item \label{item:5-exceptional} $5 \mid \twerr(a,b,c)$ if and only if $a \equiv 7b \md 25$ and $5
    \mid c$.
  \end{enumerate}
  We say that $(a,b,c)$ is \cdef{2-exceptional} when \ref{item:2-exceptional}
  holds, and \cdef{5-exceptional} when \ref{item:5-exceptional} holds.
\end{lemma}
\begin{proof}
  \hfill
  \begin{enumerate}[label=(\alph*)]
  \item If $2 \mid \twerr(a,b,c)$, then $4 \mid A$. This implies that $a$ and
    $c$ have the same parity. Since $c \equiv 1 \md 4$, $a$ must be odd. But
    since $a$ and $b$ have oposite parity, we conclude that $2 \mid b$.
    Conversely, suppose that $2 \mid b$. This implies that $a$ is odd. Since $9
    \mid B$ always, we only need to show that $4 \mid A$. But this is visibly
    true once we know that $a$ is odd.
  \item If $5 \mid \twerr(a,b,c)$, then $A \equiv 0 \md 25$. This implies that
    $a \equiv 7b \md 25$. From this congruence, we see that $c^4 = a^2 +
    b^2 = 50b^2 \equiv 0 \md 25$, so that $5 \mid c$. Conversely, the
    congruence $a \equiv 7b \md 25$ implies that $A \equiv 0 \md 25$ and $B/(8c)
    \equiv 0 \md 25$, which is enough to conclude that $5 \mid \twerr(a,b,c)$.
  \end{enumerate}
\end{proof}
\santi{Can we characterize all five exceptional triples with a simple criterion
in terms of Gaussian integers? Note that if $5 \mid c$, we can take $\pi \in
\{1 + 2i, 1-2i\}$, and then every for every $k > 0$ we can produce a twist
minimal triple of the form $(\Re(i^n\pi^{4k}), \Im(i^n\pi^{4k}), 5^k)$. A bunch of
these turn out to be $5$-exceptional, but not all of them. (See the Magma code
commented here)}.
\spark{Yes, please refer to the comments I left in the next question, or Lemma 7.7 of this updated document.}

% ZZ := Integers();
% t := PolynomialRing(Integers()).1;
% K<i> := NumberField(t^2 + 1);

% candidates := [];
% for k := 1 to 10 do
%     for pi in {1 + 2*i, 1-2*i} do;
%         c := 5^k;
%         for n :=0 to 3 do
%             alpha := i^n*pi^(4*k);
%             a := ZZ!(alpha[1]);
%             b := ZZ!(alpha[2]);
%             if IsDivisibleBy(7*a + b,25) then Append(~candidates, [a,b,c]); end if;
%         end for;
%     end for;
% end for;

\spark{Please refer to the updated Magma code commented here. This code will be relevant to Lemma 7.7}
% ZZ := Integers();
% t := PolynomialRing(Integers()).1;
% K<i> := NumberField(t^2 + 1);

% candidates := [];
% for k := 1 to 10 do
%     for pi in {1 + 2*i, 1-2*i} do;
%         c := 5^k;
%         for n :=0 to 3 do
%             alpha := i^n*pi^(4*k);
%             a := ZZ!(alpha[1]);
%             b := ZZ!(alpha[2]);
%             if IsDivisibleBy(7*a + b,5^2) then Append(~candidates, [a,b,c]); end if;
%             beta := i^n*pi^(4*k-2);
%             a := 5*ZZ!(beta[1]);
%             b := 5*ZZ!(beta[2]);
%             if IsDivisibleBy(7*a + b,5^2) then Append(~candidates, [a,b,c]); end if;
%         end for;
%     end for;
% end for;
% print(candidates);

\begin{lemma}
  \label{lemma:4-nmid-twerr}
  Let $(a,b,c)$ be a twist-minimal triple. Then
  $\twerr(a,b,c)$ is not divisible by $4$.
\end{lemma}
\begin{proof}
  Suppose that $4 \mid \twerr(a,b,c)$. By definition, this means that $4^2 \mid
  A$ and $4^3 \mid B$. From \Cref{eq:A,eq:B}, we deduce that
  \begin{align*}
    123a + 114b + 125c^2 &\equiv 0 \md 8, \\
    2502a + 261b + 2500c^2 &\equiv 0 \md 8.
  \end{align*}
  Since $c \equiv 1 \md 4$ (\Cref{lemma:twist-min-triples}), we have that
  $c^2 \equiv 1 \md 8$, and the above congruences simplify to:
    \begin{align*}
    3a + 2b + 5 &\equiv 0 \md 8, \\
    6a + 5b + 4 &\equiv 0 \md 8.
    \end{align*}
    These imply that $b \equiv 6 \md 8$, and $a^2 + b^2 = c^4$ imlpies that
    $a^2 \equiv 5 \md 8$. But this contradicts the fact that $5$ is not a
    square modulo $8$.
  \end{proof}

  \santi{Sun Woo, according to Lemma 6.3 in the Overleaf document, we should
    have that for a twist minimal triple $(a,b,c)$, the only possibilities for
    $\twerr(a,b,c)$ are $10, 5, 2, 1$. So far, I agree that when $(a,b,c)$ is
    not $5$-exceptional, then $\twerr(a,b,c) = 2, 1$. But after some struggle,
    I found an exceptional twist minimal triple with twist error equal to $25$:
    take $(a,b,c) = (-527, -336, 25)$. From the examples I have computed, $25$
    seems to be the greatest power of $5$ dividing $\twerr(a,b,c)$. Can we show
    that $5^3$ does not divide $\twerr$?}

\spark{Thank you very much for pointing out the mistake, Santi. The two lemmas below outlines the classification of 5-exceptional twist-minimal triples. Maybe Lemma 7.6 is redundant, as Lemma 7.7 already provides the proof that $\twerr(a,b,c)$ is not divisible by $5^4$.}

\begin{lemma} \label{lemma:625-nmid-twerr}
    Let (a,b,c) be a twist-minimal triple. Then $\twerr(a,b,c)$ is not divisible by $5^4$.
\end{lemma}
\begin{proof}
    Suppose that $5^4 \mid \twerr(a,b,c)$. By definition, this implies that $5^8 \mid A$ and $5^{12} \mid B$. Using \Cref{eq:A,eq:B}, we deduce that
    \begin{align*}
        123a + 114b + 125c^2 &\equiv 0 \text{ mod } 5^8 \\
        c(2502a + 261b + 2500c^2) &\equiv 0 \text{ mod } 5^{12}.
    \end{align*}

    We divide into 3 cases depending on the valuation of $c$ with respect to $5$.

\medskip

    \textbf{Case 1}: Suppose $5^3 \mid c$. Then \Cref{eq:A} implies
    \begin{equation*}
        41 a + 38b \equiv 0 \text{ mod } 5^8.
    \end{equation*}
    Multiplying both sides of the equation by 238186, we obtain
    \begin{equation*}
        a + 66693b \equiv 0 \text{ mod } 5^8
    \end{equation*}
    which is equivalent to
    \begin{equation*}
        a^2 \equiv 299999 b^2 \text{ mod } 5^8 
    \end{equation*}
    Using the equation $a^2 + b^2 = c^4$, we obtain
    \begin{equation*}
        a^2 + b^2 \equiv 300000a^2 = 2^5 \cdot 3 \cdot 5^5 \cdot a^2 \equiv 0 \text{ mod } 5^8
    \end{equation*}
    This implies that $a \equiv b \equiv 0 \text{ mod } 5^2$, which contradicts that $(a,b,c)$ is twist minimal.

    \medskip 

    \textbf{Case 2}: Suppose that $5^2 \mid c$ but $5^3 \nmid c$. Then \Cref{eq:A,eq:B} imply
    \begin{align*}
        41a + 38b &\equiv 0 \text{ mod } 5^7 \\
        278a + 29b &\equiv 0 \text{ mod } 5^7
    \end{align*}
    Multiply the first equation by 26727 to obtain
    \begin{equation*}
        2057a + b \equiv 0 \text{ mod } 5^7
    \end{equation*}
    Substitute into the second equation to obtain
    \begin{equation*}
        18750a = 2 \cdot 3 \cdot 5^5 a \equiv 0 \text{ mod } 5^7
    \end{equation*}
    This implies that $a \equiv b \equiv 0 \text{ mod } 5^2$, which contradicts that $(a,b,c)$ is twist minimal.

    \medskip 

    \textbf{Case 3}: Suppose that $5 \mid c$ but $5^2 \nmid c$. Then \Cref{eq:A} implies
    \begin{equation*}
        41a + 38b \equiv 0 \text{ mod } 5^5.
    \end{equation*}
    Multiply the equation by 1727 to obtain
    \begin{equation*}
        2057a + b \equiv 0 \text{ mod } 5^5,
    \end{equation*}
    which implies
    \begin{equation*}
        3124a^2 \equiv b^2 \text{ mod } 5^5.
    \end{equation*}
    Substitute the equation to $a^2 + b^2 = c^4$ to obtain
    \begin{equation*}
        a^2 + b^2 \equiv 3125 a^2 \equiv 5^5 a^2 \equiv 0 \equiv c^4 \text{ mod } 5^5.
    \end{equation*}
    This implies that $5^2 \mid c$, a contradiction.

    \medskip 

    \textbf{Case 4}: Suppose that $5 \nmid c$. Then \Cref{eq:B} implies
    \begin{equation*}
        278a + 29b \equiv 0 \text{ mod } 5^4.
    \end{equation*}
    Multiply the equation by 194 to obtain
    \begin{equation*}
        182a + b \equiv 0 \text{ mod } 5^4,
    \end{equation*}
    which implies
    \begin{equation*}
        624a^2 \equiv b^2 \text{ mod } 5^4.
    \end{equation*}
    Substitute the equation to $a^2 + b^2 = c^4$ to obtain
    \begin{equation*}
        a^2 + b^2 \equiv 625 a^2 \equiv 5^4 a^2 \equiv 0 \equiv c^4 \text{ mod } 5^4.
    \end{equation*}
    This implies that $5 \mid c$, a contradiction.
\end{proof}

\begin{lemma}
    Let (a,b,c) be a 5-exceptional twist-minimal triple. Let $l$ be the valuation of $\twerr(a,b,c)$ with respect to $5$. Note that $1 \leq l \leq 3$ by Lemma \ref{lemma:625-nmid-twerr}. Let $d \in \mathbb{Z}[i]$ be any Gaussian integer coprime to $(1+2i)$ and $(1-2i)$ whose prime factors are all not equivalent to 3 modulo 4 and whose norm is an integral $4$-th power. 
    \begin{itemize}
        \item Suppose $l = 1$. Then the twist minimal triple $(a,b,c)$ is of the form
        \begin{equation}
            (a,b,c) = \begin{cases} \left(\Re(5 \cdot (1-2i)^2 \cdot d), \Im(5 \cdot (1-2i)^2 \cdot d), 5 \cdot \sqrt{d \overline{d}} \right) \\
            \left(\Re((1-2i)^4 \cdot d), \Im((1-2i)^4 \cdot d), 5 \cdot \sqrt[4]{d \overline{d}} \right)
            \end{cases},
        \end{equation}
        \item Suppose $l = 2$. Then the twist minimal triple $(a,b,c)$ is of the form
        \begin{equation}
            (a,b,c) = \left(\Re((1-2i)^{4k} \cdot d), \Im((1-2i)^{4k} \cdot d), 5^k \cdot \sqrt[4]{d \overline{d}} \right)
        \end{equation}
        for any positive integer $k \geq 2$.
        \item Suppose $l = 3$. Then the twist minimal triple $(a,b,c)$ is of the form
        \begin{equation}
            (a,b,c) = \left(\Re(5 \cdot (1-2i)^{4k-2} \cdot d), \Im(5 \cdot (1-2i)^{4k-2} \cdot d), 5^k \cdot \sqrt[4]{d \overline{d}} \right)
        \end{equation}
        for any positive integer $k \geq 2$.
    \end{itemize}
\end{lemma}
\begin{proof}
    We recall the parts of \Cref{eq:A,eq:B} excluding the terms containing $c^2$.
    \begin{align*}
        &123a + 114b = 3(41a + 38b) \\
        &2502a + 261b = 9(278a + 29b)
    \end{align*}
    Observe that
    \begin{align*}
        (1+2i)^5 &= 41 - 38i\\
        (1+2i)^7 &= 29 + 278i
    \end{align*}
    Multiply the above equations by $a+bi$ to obtain
    \begin{align*}
        (1+2i)^5 \cdot (a+bi) &= (41-38i) \cdot (a+bi) = (41a + 38b) + i(41b - 38a) \\
        (1+2i)^7 \cdot (a+bi) &= (29+278i) \cdot (a+bi) = (29a -278b) + i(278a + 29b)
    \end{align*}
    We note that the valuations of $41a + 38b$ and $41b-38a$ with respect to $5$ are equal. To see this, we take norm on both sides of the equation to obtain
    \begin{equation*}
        5^5 \cdot c^4 = 5^5 \cdot (a^2 + b^2) = (41a + 38b)^2 + (41b-38a)^2
    \end{equation*}
    If $41a + 38b$ and $41b-38a$ have different valuations with respect to $5$, then the valuation of the RHS of the equation is even, which contradicts the fact that the valuation of the LHS of the equation is odd. The analogous reasoning shows that the valuation of $29a - 278b$ ad $278a + 29b$ with respect to $5$ are identical.
    
    We obtain that the valuations of $41a + 38b$ and $278a + 29b$ with respect to $5$ are determined by the valuation of $a+bi$ with respect to $1-2i$. Because $(a,b,c)$ is a twist-minimal triple, $a^2 + b^2 = c^4$, and $5 \mid c$ from \ref{lemma:exceptional-triples}, the 5-exceptional triples $(a,b,c)$ satisfy
    \begin{align*}
        a+bi = \begin{cases}
            (1-2i)^{4k} \cdot d \\
            5 \cdot (1-2i)^{4k-2} \cdot d
        \end{cases}
    \end{align*}
    for any positive integer $k \geq 1$ and $d \in \mathbb{Z}[i]$ an element that does not lie in the ideals $(1+2i)$ and $(1-2i)$ such that $d$ is not divisible by primes equivalent to $3$ mod $4$ and whose norm is an integral $4$-th power. We use the abbreviation $v_5(\cdot)$ to denote the valuation of an element $\cdot$ with respect to $5$. Then substituting the above expression to \Cref{eq:A} gives
    \begin{equation} \label{eq:5-val-A}
        v_5(A(a,b,c)) = \begin{cases}
            3 &\text{ if } a+bi = 5 \cdot (1-2i)^2 \cdot d \\
            4 &\text{ if } a+bi = (1-2i)^4 \cdot d \\
            5 &\text{ if } a+bi = (1-2i)^{4k} \cdot d \text{ for } k \geq 2 \\
            6 &\text{ if } a+bi = 5 \cdot (1-2i)^{4k-2} \cdot d \text{ for } k \geq 2
        \end{cases}
    \end{equation}
    Likewise, substituting the above expression to \Cref{eq:B} gives
    \begin{equation} \label{eq:5-val-B}
        v_5(B(a,b,c)) = \begin{cases}
            4 &\text{ if } a+bi = 5 \cdot (1-2i)^2 \cdot d \\
            5 &\text{ if } a+bi = (1-2i)^4 \cdot d \\
            k + 7 &\text{ if } a+bi = (1-2i)^{4k} \cdot d \text{ for } k \geq 2 \\
            k + 7 &\text{ if } a+bi = 5 \cdot (1-2i)^{4k-2} \cdot d \text{ for } k \geq 2
        \end{cases}
    \end{equation}
    Combining \Cref{eq:5-val-A,eq:5-val-B}, we obtain
    \begin{align*}
        v_5(\twerr(a,b,c)) = 1 &\iff a+bi = 5 \cdot (1-2i)^2 \cdot d \text{ or } = (1-2i)^4 \cdot d \\
        v_5(\twerr(a,b,c)) = 2 &\iff a+bi = (1-2i)^{4k} \cdot d \text{ for } k \geq 2 \\
        v_5(\twerr(a,b,c)) = 3 &\iff a+bi = 5 \cdot (1-2i)^{4k-2} \cdot d \text{ for } k \geq 2
    \end{align*}
\end{proof}

\spark{For example, suppose that $d=1$. Then Lemma 7.7 shows that if $c = 5$, then there are $8$ twist-minimal triples such that $5 \mid \twerr(a,b,c)$ and $25 \nmid \twerr(a,b,c)$. These are: 
\begin{equation*}
    a+bi \in \{-15-20i, 20-15i, 15 + 20i, -20+15i, -7 + 24i, -24-7i, 7 - 24i, 24 + 7i\}
\end{equation*}
These 8 elements correspond to the following triples:
\begin{equation*}
    (a,b) \in \{(-15,-20),(20,-15),(15,20),(-20,15),(-7,24),(-24,-7),(7,-24),(24,7)\}
\end{equation*}
This agrees with the result of the updated magma code commented below.
}

% ZZ := Integers();
% t := PolynomialRing(Integers()).1;
% K<i> := NumberField(t^2 + 1);

% candidates := [];
% for k := 1 to 10 do
%     for pi in {1 + 2*i, 1-2*i} do;
%         c := 5^k;
%         for n :=0 to 3 do
%             alpha := i^n*pi^(4*k);
%             a := ZZ!(alpha[1]);
%             b := ZZ!(alpha[2]);
%             if IsDivisibleBy(7*a + b,5^2) then Append(~candidates, [a,b,c]); end if;
%             beta := i^n*pi^(4*k-2);
%             a := 5*ZZ!(beta[1]);
%             b := 5*ZZ!(beta[2]);
%             if IsDivisibleBy(7*a + b,5^2) then Append(~candidates, [a,b,c]); end if;
%         end for;
%     end for;
% end for;
% print(candidates);

\spark{Here's a question we would have to figure out. Based on the two previous lemmas, the possible values of $\twerr(a,b,c)$ are: $1, 2, 5, 10, 25, 50, 125, 250$. We would need to verify whether the twist-minimal triples with such values of $\twerr(a,b,c)$ give rise to non-isomorphic families of elliptic curves, up to quadratic twists over $\mathbb{Q}$.}

\clearpage
\bibliography{refs.bib}{} \bibliographystyle{amsalpha}
\end{document}
