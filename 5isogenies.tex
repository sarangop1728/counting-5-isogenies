\documentclass[12pt]{amsart}
% PREAMBLE
\usepackage{mathscinet}
\usepackage[utf8]{inputenc} % encoding
\usepackage[dvipsnames, table]{xcolor} % colors
\usepackage{color} % colors
\usepackage{tikz} % code figures
\usepackage{tikz-cd} % commutative diagrams
\usetikzlibrary{matrix,arrows,decorations.pathmorphing} % special decorations for tikzmain.tex
\usepackage{amssymb, amsmath, mathtools, amsfonts, amsthm} % ams    
\usepackage{stmaryrd}
\usepackage{thmtools,
  thm-restate} % restate theorems and other options
\declaretheorem[numberwithin=section, name = Theorem]{mainthm}
% \usepackage{pdfpages} % inclusion of external PDF documents in LATEX
\usepackage{graphicx} % optional arguments for \includegraphics
\usepackage{fullpage} % sets margins to be either 1 inch or 1.5 cm
\usepackage{caption} % customize the captions in floating environments
\usepackage{latexsym} % for \lhd and \unlhd (and others)
\usepackage{enumitem} % customize items in lists
\usepackage[toc,page]{appendix} % formatting the titles of appendices
\usepackage{multicol} % typeset text in multiple columns
\usepackage{hyperref}
\definecolor{headercolor}{RGB}{255,255,240}
\definecolor{mylinkcolor}{RGB}{0,0,255}
\definecolor{mycitecolor}{RGB}{169,169,169}
\definecolor{myurlcolor}{RGB}{255,20,147}
% see https://latexcolor.com/ for more options
\definecolor{santicolor}{rgb}{0.0, 0.8, 0.6}
\hypersetup{colorlinks=true,
urlcolor=myurlcolor,
citecolor=mycitecolor,
linkcolor=mylinkcolor,
linktoc=page,
breaklinks=true}
\usepackage{url} 
\urlstyle{same}

\usepackage{cleveref}
% \usepackage{chngcntr}
% \counterwithin{figure}{section} %CHANGES FIGURE COUNTERS TO SECTION
% \counterwithin{table}{section}

\usepackage[numbers]{natbib} % bibliography style
\usepackage{setspace}
\setlength{\bibsep}{0.0pt} % compact bibliography
\usepackage{indentfirst} % to indent the first paragraph of a section
\usepackage{listings} % source-code printing
\usepackage[mathscr]{eucal} % nice caligraphy for sheaves and categories
\usepackage{manfnt} % street sign
\usepackage{longtable} % tables that continue on the next page
\usepackage{colonequals} % correct way to write :=
\usepackage{nicefrac} % typeset in-line fractions in a "nice" way. \nicefrac[\mathcal]{numerator}{denominator}
\usepackage{wrapfig}
% \usepackage{draftwatermark} % watermarks for drafts
% \SetWatermarkText{DRAFT} \SetWatermarkScale{1}
%\usepackage{bbold} % to use \mathbb{1}
\usepackage{tkz-euclide} % euclidean geometry
\usepackage{caption} % figures and subfigures
\usepackage{subcaption} % figures and subfigures
\usepackage{float}
\usepackage{soul} % to strikethrough words: \st{}
\usepackage{pgfplots}
\pgfplotsset{width=10cm,compat=1.9} 

% \usepackage{mathpazo}
% \usepackage{lineno} %numering lines
% \linenumbers

\usepackage{listings}
\definecolor{codegreen}{rgb}{0,0.6,0}
\definecolor{codegray}{rgb}{0.5,0.5,0.5}
\definecolor{codepurple}{rgb}{0.58,0,0.82}
\definecolor{backcolour}{rgb}{0.95,0.95,0.92}

\lstdefinestyle{mystyle}{ backgroundcolor=\color{backcolour},   commentstyle=\color
 {codegreen}, keywordstyle=\color{magenta}, numberstyle=\tiny\color{codegray}, stringstyle=\color
 {codepurple}, basicstyle=\ttfamily\footnotesize, breakatwhitespace=false,         breaklines=true,
 captionpos=b,                    keepspaces=true,                 numbers=left, numbersep=5pt,
 showspaces=false,                showstringspaces=false, showtabs=false, tabsize=2 }

\lstset{style=mystyle, language=[Objective]C}


% THEOREM ENVIRONMENTS.

\newcounter{counter}[section] % global counter / increases with subsection.
\renewcommand{\thecounter}{\thesection.\arabic{counter}}


\newenvironment{warning}{\vspace{.5em} \textdbend \textbf{Warning.}}{
  \vspace*{1em}}
\numberwithin{equation}{section} % numbering according to section
\newtheorem{thmx}{Theorem}
\renewcommand{\thethmx}{\Alph{thmx}} % alphabetic numbering
\newtheorem{corox}{Corollary} \renewcommand{\thecorox}{\Alph{corox}}

% \numberwithin{equation}{subsection} % numbering according to subsection
\newtheorem{theorem}[counter]{Theorem}
\newtheorem{lemma}[counter]{Lemma}
\newtheorem{corollary}[counter]{Corollary}
\newtheorem{conjecture}[counter]{Conjecture}
\newtheorem{proposition}[counter]{Proposition}
\newtheorem{question}[counter]{Question}


\theoremstyle{definition} \newtheorem{definition}[counter]{Definition}
\newtheorem{example}[counter]{Example} % examples
\newtheorem{observations}[counter]{Observations}
\newtheorem*{answer}{Answer} \newtheorem*{fact}{N.B}
\newtheorem{assume}[counter]{Assumption} \newtheorem*{claim}{Claim}
\newtheorem*{List}{\underline{List}} \newtheorem*{apply}{Application}


\theoremstyle{remark} \newtheorem{nonexam}[counter]{Non-example}
\newtheorem*{notation}{Notation}
\newtheorem{remark}[counter]{Remark}
\newtheorem{remarks}[counter]{Remarks}




% COMMAND SHORTCUTS.
  
% Random.
\newcommand{\R}{\mathcal{R}} % region
\newcommand{\RR}{\mathbb{R}} % for Real numbers
\newcommand{\ZZ}{\mathbb{Z}} % for Integers
\newcommand{\QQ}{\mathbb{Q}} % for Rationals
\newcommand{\CC}{\mathbb{C}} % for Complex numbers
\newcommand{\OO}{\mathscr{O}} % for ring of integers
\renewcommand{\AA}{\mathbf{A}} % affine space
\newcommand{\PP}{\mathbf{P}} % projective space
\newcommand{\Qp}{\mathbf{Q}_p} % for p-adics
\newcommand{\Qbar}{\bar{\QQ}} % for algebraic numbers
\newcommand{\XX}{\mathscr{X}} % mathscr X
\newcommand{\XOF}{\XX_0(5)} % stacky X0(5)
\newcommand{\XOFt}{\XX_0^{tw}(5)} %rigidification of X0(5)
\newcommand{\YY}{\mathscr{Y}} % mathscr Y
\newcommand{\FF}{\mathscr{F}} % mathscr Y
\newcommand{\GG}{\mathscr{G}} 
\newcommand{\sW}{\mathscr{W}} 
\newcommand{\sE}{\mathscr{E}} 
\newcommand{\Fp}{\mathbf{F}_p} % field with p elements
\newcommand{\Fq}{\mathbf{F}_q} % field with q elements
\newcommand{\Pone}{\mathbf{P}^1} % for projective line
\newcommand{\Aone}{\mathbf{A}^1} % affine line
\newcommand{\degr}{\operatorname{deg }} % for degree
\newcommand{\md}{\text{ mod }} % mod without parenthesis
\newcommand{\inv}{^{-1}} % inverse
\newcommand{\unit}{^{\times}} % unit groupo functor
\newcommand{\dd}{\,\mathrm{d}} % "d" of differential
\newcommand{\wt}[1]{\widetilde{#1}} % wide tilde
\newcommand{\paren}[1]{\left( #1 \right)} % parenthesis
\newcommand{\brk}[1]{\left\lbrace #1 \right\rbrace} % brackets
\newcommand{\sqbrk}[1]{\left[ #1 \right]} % square brackets
\newcommand{\abs}[1]{\left| #1 \right|} % absolute value
\newcommand{\ceil}[1]{\lceil #1 \rceil} % ceiling funtion
\newcommand{\ZmodnZ}[1]{\ZZ / #1 \ZZ} % Z/nZ
\newcommand{\floor}[1]{\left\lfloor #1 \right\rfloor} % floor function
\newcommand{\frob}{\mathrm{Frob}} % frobenius
\newcommand{\G}[1][\QQ]{\mathrm{Gal}_{#1}} % absolute Galois group
\newcommand{\pabs}[1]{\left| #1 \right|_p} % p-adic absolute value
\newcommand{\ideal}[1]{\langle #1 \rangle} % ideal
\newcommand{\cdef}[1]{\textsf{#1}} % highlight words in definitions
\renewcommand{\Re}{\mathrm{Re}} % real part
\renewcommand{\Im}{\mathrm{Im}} % imaginary part
\newcommand\doublelong[2]{\mathbin{\xymatrix{{}\ar@<3pt>[r]^{#1}
\ar@<-3pt>[r]_{#2}&}}} % command needed for rigidification
\newcommand{\thickslash}{\mathbin{\!\!\pmb{\fatslash}}} % symbol for rigidification




\newcommand{\ol}{\overline}
% Mathmode.
\newcommand{\mbb}[1]{\mathbb{#1}} % mathbb
\newcommand{\mbf}[1]{\mathbf{#1}} % mathbf
\newcommand{\msf}[1]{\mathsf{#1}} % mathsf
\newcommand{\mscr}[1]{\mathscr{#1}} % mathscr
\newcommand{\mst}{\text{ such that }} % "such that" text in mathmode
\newcommand{\mwith}{\text{ with }} % "with" in mathmode
\newcommand{\mand}{\text{ and }} % "and" in mathmode
\newcommand{\mfor}{\text{ for }} % "for" in mathmode
\newcommand{\mforall}{ \text{ for all }} % "for all" in mathmode
\newcommand{\mof}{\text{ of }} % "of" in mathmode
\newcommand{\mif}{\text{ if }} % "if" in mathmode
\newcommand{\mby}{\text{ by }} % "by" in mathmode
\newcommand{\mor}{\text{ or }} % "or" in mathmode
\newcommand{\mto}{\text{ to }} % "to" in mathmode
\newcommand{\motherwise}{\text{ otherwise }} % "otherwise" in mathmode

% Groups.
\newcommand{\GL}{\msf{GL}} % General Linear
\newcommand{\SL}{\msf{SL}} % Special Linear
\newcommand{\GSp}{\msf{GSp}} % General Simplectic
\newcommand{\Sp}{\msf{Sp}} % Simplectic
\newcommand{\UU}{\msf{U}} % Unitary
\newcommand{\Gm}{\mathbb{G}_m} % Multiplicative
\newcommand{\Magma}{\texttt{Magma}}
\newcommand{\cN}{\mathcal{N}}
\newcommand{\cP}{\mathcal{P}}
\newcommand{\stkC}{\mathscr{C}} % Stacky C
\newcommand{\stkP}{\mathscr{P}} % Stacky P




% Remove weird space before "\left".
\let\originalleft\left \let\originalright\right
\renewcommand{\left}{\mathopen{}\mathclose\bgroup\originalleft}
  \renewcommand{\right}{\aftergroup\egroup\originalright}

% COMMENTS IN PDF.
\newcommand{\sapedit}[1]{{\color{cyan} \sf #1}}
\newcommand{\santi}[1]{{\color{cyan} \sf
    $\spadesuit$ Santi: #1}}
\newcommand{\margSAP}[1]{\normalsize{{\color{cyan}\footnote{\color{cyan}#1}}{\marginpar[{\color{cyan}\hfill\tiny\thefootnote$\rightarrow$}]{{\color{cyan}$\leftarrow$\tiny\thefootnote}}}}}
\newcommand{\Santi}[1]{\margSAP{(Santiago) #1}}

% COMMENTS IN PDF.
\newcommand{\chanedit}[1]{{\color{teal} \sf #1}}
\newcommand{\chan}[1]{{\color{teal} \sf
    $\clubsuit\clubsuit\clubsuit$ Changho: [#1]}}
\newcommand{\margCHAN}[1]{\normalsize{{\color{teal}\footnote{\color{teal}#1}}{\marginpar[{\color{teal}\hfill\tiny\thefootnote$\rightarrow$}]{{\color{teal}$\leftarrow$\tiny\thefootnote}}}}}
\newcommand{\Changho}[1]{\margCHAN{(Changho) #1}}

\definecolor{orchid}{rgb}{0.85, 0.44, 0.84}

% COMMENTS IN PDF.
\newcommand{\oanaedit}[1]{{\color{orchid} \sf #1}}
\newcommand{\oana}[1]{{\color{orchid} \sf
    $\clubsuit\clubsuit\clubsuit$ Oana: [#1]}}
\newcommand{\margOANA}[1]{\normalsize{{\color{orchid}\footnote{\color{orchid}#1}}{\marginpar[{\color{orchid}\hfill\tiny\thefootnote$\rightarrow$}]{{\color{orchid}$\leftarrow$\tiny\thefootnote}}}}}
\newcommand{\Oana}[1]{\margOANA{(Oana) #1}}

% COMMENTS IN PDF. (Sun Woo)
\newcommand{\sparkedit}[1]{{\color{olive} \sf #1}}
\newcommand{\spark}[1]{{\color{olive} \sf
    $\clubsuit\clubsuit\clubsuit$ Sun Woo: [#1]}}
\newcommand{\margSPARK}[1]{\normalsize{{\color{olive}\footnote{\color{olive}#1}}{\marginpar[{\color{olive}\hfill\tiny\thefootnote$\rightarrow$}]{{\color{olive}$\leftarrow$\tiny\thefootnote}}}}}
\newcommand{\Sunwoo}[1]{\margSPARK{(Sun Woo) #1}}

% OPERATORS
\DeclareMathOperator{\height}{Ht} % height
\DeclareMathOperator{\len}{length} % length
\DeclareMathOperator{\nrm}{N} % norm
\DeclareMathOperator{\Nm}{Nm} % norm
\DeclareMathOperator{\disc}{disc} % discriminant
\DeclareMathOperator{\ord}{ord} % order
\DeclareMathOperator{\car}{char} % characteristic
\DeclareMathOperator{\Aut}{Aut} % automorphism
\DeclareMathOperator{\Gal}{Gal} % galois
\DeclareMathOperator{\id}{id} % identity
\DeclareMathOperator{\Frac}{Frac} % fraction field
\DeclareMathOperator{\splt}{Splt} % split
\DeclareMathOperator{\Br}{Br} % Brauer group
\DeclareMathOperator{\codim}{codim} % codimension
\DeclareMathOperator{\SFab}{Stab} % stabilizer
\DeclareMathOperator{\Tr}{Tr} % trace
\DeclareMathOperator{\Hom}{Hom} % Hom set
\DeclareMathOperator{\Mor}{Mor} % Morphism set
\DeclareMathOperator{\rank}{rk} % rank
\DeclareMathOperator{\img}{Im} % Image
\DeclareMathOperator{\im}{im} % image
\DeclareMathOperator{\Pic}{Pic} % Picard group
\DeclareMathOperator{\Sylow}{Syl} % Sylow
\DeclareMathOperator{\Jac}{Jac} % Jacobian
\DeclareMathOperator{\Spec}{Spec} % Spec
\DeclareMathOperator{\mSpec}{mSpec} % maximal Spec
\DeclareMathOperator{\lcm}{lcm} % least common multiple
\DeclareMathOperator{\sgn}{sgn} % sign
\DeclareMathOperator{\Sym}{Sym} % symmetries
\DeclareMathOperator{\coker}{coker} % cokernel
\DeclareMathOperator{\Div}{Div} % Divisor group
\DeclareMathOperator{\Proj}{Proj} % Proj
\DeclareMathOperator{\supp}{supp} % support
\DeclareMathOperator{\res}{res} % residue
\DeclareMathOperator{\vol}{vol} % volume
\DeclareMathOperator{\Span}{Span} % span
\DeclareMathOperator{\ev}{ev} % evaluation map
\DeclareMathOperator{\dR}{dR} % de Rham
\DeclareMathOperator{\trdeg}{tr.deg} % transcendence degree
\DeclareMathOperator{\Res}{Res} % Residue
\DeclareMathOperator{\Log}{Log} % Logarithm
\DeclareMathOperator{\sheafHom}{\mathscr{H}\text{\kern -1pt
    {om}}} % Sheaf Hom
\DeclareMathOperator{\sProj}{\mathscr{P}\text{\kern -1pt
    {roj}}} % Sheaf Hom
    
\DeclareMathOperator{\mind}{md} % minimality defect
\DeclareMathOperator{\twmind}{tmd} % twist minimality defect
\DeclareMathOperator{\Ht}{Ht} % height
\DeclareMathOperator{\tw}{tw} % twist
\DeclareMathOperator{\twht}{twHt} % twist height
\DeclareMathOperator{\twerr}{twerr} % twist error
\DeclareMathOperator{\Fermat}{Fermat} % Fermat triple associated to a Gaussian
                                      % integer



% OPENING
\title{Counting minimal integral points on $x^2+y^2 = z^4$ and 5-isogenies of elliptic curves over the rationals}


\author{Santiago Arango-Piñeros} \address{Department of Mathematics,
  Emory University, Atlanta, GA 30322, USA}
\email{santiago.arango@emory.edu}
\urladdr{\url{https://sarangop1728.github.io/}}

\author{Changho Han}
\address{Pure Mathematics, University of Waterloo, Waterloo, ON N2L 3G1, Canada}
\email{changho.han@uwaterloo.ca}
\urladdr{\url{https://hanchangho.github.io/}}

\author{Oana Padurariu} \address{Department of Mathematics and Statistics,
  Boston University}
\email{oana.adascalitei@yahoo.com}
\urladdr{\url{https://sites.google.com/view/oana-padurariu/home}}
\author{Sun Woo Park} \address{Department of Mathematics, University of Wisconsin-Madison, Madison, WI 53703, USA}
\email{spark483@wisc.edu}
\urladdr{\url{https://sites.google.com/wisc.edu/spark483}}

\begin{document}

\begin{abstract}
  We show that elliptic curves over the rationals equiped with a
  $5$-isogeny can be parametrized by minimal integral solutions to the
  Fermat equation $x^2 + y^2 = z^4$ . We use this parametrization to
  prove that the number of $5$-isogenies with naive height bounded by
  $T$ is $\sim C\cdot T^{1/6}(\log T)^2$ for some explicitly
  computable constant $C>0$. This completes the asymptotic count of
  rational points on the genus zero modular curves $X_0(N)$.
\end{abstract}

\maketitle


\section{Introduction}

\section{Analytic ingredients}\label{sec:analytic-ingredients}
\subsection{Tauberian theorems}\label{sec:tauberian-theorems} Estimates of the partial sums of a Dirichlet series can be obtained from its analytic properties via Tauberian theory. See Theorem 1.1 and Remark 1 in \cite{Murty2024} for a complete proof of the following result.

\begin{theorem}
    \label{thm:tauberian1}
    Let $G(s) = \sum_{n=1}^\infty g(n)n^{-s}$ be a Dirichlet series with
    non-negative coefficients, and suppose that $G(s) = \zeta(s)^kh(s)$ for
    some positive integer $k$, and $h(s)$ a Dirichlet series absolutely
    convergent in $\Re(s) \geq 1$, with $h(1)\neq 0$. Then, as $X \to \infty$,
    \begin{equation*}
        \sum_{n\leq X} g(n) = \dfrac{g_{1}}{(k-1)!}X(\log X)^{k-1} + O(X(\log X)^{k-2}),
    \end{equation*}
    where $g_{1}$ is the residue of $G(s)$ at $s = 1$. \\
    Furthermore, if $G(s)$ has Laurent series expansion
    \begin{equation*}
        G(s+1) = \dfrac{g_{k}}{s^k} + \dfrac{g_{k-1}}{s^{k-1}} + \cdots + g_0 + O_m(|s|^m),
    \end{equation*}
    as $s \to 0^+$, for some $g_i \in \CC$ and $m \in \ZZ_{>0}$, then
    \begin{equation*}
        \sum_{n\leq X} g(n) = X\sum_{j = 1}^k
        \dfrac{\lambda_{k-j}}{(k-j)!}(\log X)^{k-j}
        + O_m\left( \dfrac{X}{(\log X)^m}\right),
    \end{equation*}
    where 
    \begin{equation*}
        \lambda_{k-j} = \sum_{i=0}^{j-1}(-1)^{j-1-i}g_{k-i}.
    \end{equation*}
\end{theorem}

A better control of the function $h(s) = G(s)/\zeta(s)^k$ can yield better
estimates on the error term. The following result is a special case of Théorème
A.1 in \cite{CL&T01}. Recall that a complex function $F(s)$ is said to be of
\cdef{finite order} on a half-plane $\Re(s) > r$ if there exists a constant
$\kappa$ such that
    \begin{equation}
        \label{eq:finite-order}
        |F(s)| = O((1+\Im(s))^\kappa),
    \end{equation}
for $\Re(s) > r$. 

\begin{theorem}
    \label{thm:tauberian2}
    Assume the hypothesis of \Cref{thm:tauberian1}, and assume that the
    follwoing conditions are met:
    \begin{enumerate}[label=(\roman*)]
    \item There exists some $1 > \delta > 0$ such that $h(s)$ admits a
      holomorphic continuation to the half-plane $\Re(s) > 1 - \delta > 0$.
        \item The function $G(s)(s-1)^k/s^{k}$ is of finite order in $\Re(s) > 1-\delta$.
    \end{enumerate}
    Then, for every $\varepsilon >0$, 
    \begin{equation*}
      \sum_{n\leq X} g(n) =
      X\sum_{j = 1}^k \dfrac{\lambda_{k-j}}{(k-j)!}(\log X)^{k-j}
      + O_\varepsilon\left(X^{1-\delta + \varepsilon}\right),
    \end{equation*}
    as $X \to \infty$.
\end{theorem}

\section{Counting twist-minimal integer points on $x^2 + y^2 = z^4$}
\label{sec:counting-twist-min-sols}

\begin{definition}
  \label{def:fermat-triple}
  Since in this article we are concerned with the generalized Fermat equation
  of signature $(2,2,4)$, we will say that a triple of integers $(a,b,c)$ is a
  \cdef{Fermat triple} if it satisfies the equation $x^2 + y^2 = z^4$.
\end{definition}

\begin{definition}[Weighted GCD, twist, and minimality]
  \label{def:weighted-stuff}
  Given a triple of positive integers $r,s,t$, the \cdef{$(r,s,t)$-weighted
  greatest common divisor} of $(a,b,c)\in \ZZ^3$ is the greatest positive
  integer $d$ such that $d^r\mid a$, $d^s \mid b$, and $d^t \mid c$. We say
  that $(a_2,b_2,c_2)$ is an \cdef{$(r,s,t)$-twist} of $(a_1,b_1,c_1)$ if
  there exists an integer $n \neq 0$ such that
  $(a_2,b_2,c_2) = (n^ra_1,n^sb_1,n^tc_1)$. A triple of integers $(a,b,c)$ is
  called \cdef{$(r,s,t)$-minimal} if the $(r,s,t)$-weighted common divisor of
  $(a,b,c)$ is equal to one. Equivalently, if there is no prime $p$ such that
  $a^r \equiv b^s \equiv c^t \equiv 0 \md p$.
\end{definition}

At the moment, we are primarly concerned with the weight $(r,s,t) = (2,2,1)$.
This choice comes from the observation that given an integral solution
$(a,b,c)$ to the generalized Fermat equation $x^2 + y^2 = z^4$, then the twist
$(n^2a,n^2b,nc)$ yields a different one for any integer $n$. By factoring out
the $(2,2,1)$-greatest common divisor of a solution, and twisting by $-1$ when
$c <0$, we can arrive at a choice of a smallest solution.

\begin{definition}[Twist-defect and twist-height]
  An integer triple $(a,b,c)$ is
  \cdef{twist-minimal} if the following conditions are satisfied:
  \begin{itemize}
   \item $a^2 + b^2 = c^4$,
   \item $(a,b,c)$ is $(2,2,1)$-minimal, and
   \item $c>0$.
   \end{itemize}
     The \cdef{twist minimality defect} of a triple $(a,b,c)$ is its
  $(2,2,1)$-greatest common divisor, and it is denoted by $\twmind(a,b,c)$. By
  definition, every integral solution $(a,b,c)$ of $x^2 + y^2 = z^4$ is a
  $(2,2,1)$-twist of the twist-minimal solution $(a/e^2, b/e^2, |c|/e)$, where
  $e = \twmind(a,b,c)$.

  Define the \cdef{twist height} of a Fermat triple $(a,b,c)$ by
  \begin{equation}
    \label{eq:twist-height}
    \twht(a,b,c) \colonequals \dfrac{|c|}{\twmind(a,b,c)}.
  \end{equation}
\end{definition}

The main result of this section is the asymptotic count of twist minimal
triples. For a real parameter $T > 0$, denote by $N_\FF(T)$ the number of twist
minimal triples with twist height less than or equal to $T$. We will prove the
following theorem.

\begin{theorem}
    \label{thm:FF-count}
    There exist explicitly computable constants $f_1, f_2 > 0$ such that for
    every $\epsilon > 0$, we have
    $$N_{\FF}(T) = f_1 T(\log T) + f_2T + O_\epsilon(T^{1/2+\epsilon}),$$
    as $T \to \infty$. Furthermore, the constant $f_1$ is given by 
    \begin{equation*}
      f_1 = \dfrac{\gamma_0\pi^2}{8}\prod_{p \, \equiv\, 3 \md 4}\paren{1-p^{-2}}^2
      \prod_{p \, \equiv\, 1 \md 4}\paren{1 -6p^{-2}+8p^{-3}-3p^{-4}},
      \end{equation*}
      where $\gamma_0$ is the Euler-Mascheroni constant.
\end{theorem}

Before diving into the proof, we record some observations about twist-minimal
triples that will be used in the argument and later on in the article.

\begin{lemma}[Characterization of twist-minimal triples]
  \label{lemma:twist-min-triples}
  Suppose that $(a,b,c)$ is a twist-minimal triple. Then, the
  following are true.
  \begin{enumerate}[label=(\alph*)]
  \item $\gcd(a,b)$ is square free.
  \item If $p \mid \gcd(a,b)$, then $p \mid c$.
  \item If $p \mid c$, then $p \equiv 1 \md 4$.
  \item $c \equiv 1 \md 4$.
  \item $a$ and $b$ have distinct parity.
  \item If $a = 0$, then $b^2 = c^2 = 1$.
  \item If $b = 0$, then $a^2 = c^2 = 1$.
  \item If $\alpha \in \ZZ[i]$, define
    $\Fermat(\alpha) \colonequals (\Re(\alpha), \Im(\alpha),
    (\alpha\ol{\alpha})^{1/4})$. Then, $\Fermat(\alpha)$ is a twist-mininmal triple if
    and only if
    \begin{itemize}
    \item every prime factor $\pi$ of $\alpha\ZZ[i]$ lies above $2$ or a prime
      $p \equiv 1 \md 4$,
      \item $\ord_p(\alpha) \leq 1$ for every rational prime $p$ dividing
        $\alpha$, and
      \item $\ord_\pi(\alpha) + \ord_{\ol{\pi}}(\alpha) = 4k$ for some
        $k \in\ZZ_{>0}$.
    \end{itemize}
    We say that $\alpha \in \ZZ[i]$ is \cdef{twist-minimal} if
    $\Fermat(\alpha)$ is a twist-minimal triple.
  \end{enumerate}
\end{lemma}
\begin{proof}
  \hfill
  \begin{enumerate}[label=(\alph*)]
  \item If $p^2$ divides both $a$ and $b$, then $p^4 \mid c^4$, which
    implies that $c$ is divisible by $p$. This contradicts the
    $(2,2,1)$-minimality of $(a,b,c)$.
  \item It follows from $a^2 + b^2 = c^4$.
  \item Suppose that $2$ divides $c$, and write $v = v_2(c) > 0$. Then,
    $2^{4v} \mid a^2 + b^2 = (a+ib)(a-ib)$ and we have the equation of ideals
    $\langle a+ib\rangle = \langle 1+i\rangle^{4v}\cdot J = \langle
    2\rangle^{2v}\cdot J$ for some ideal $J \subset \ZZ[i]$. This shows that
    $2^2$ divides $a$ and $b$, implying that $(a,b,c)$ is not $(2,2,1)$-minimal
    at $p=2$, contradiction. If $p \equiv 3 \md 4$ and $p$ divides $c$, then
    $p$ is prime in $\ZZ[i]$ and this implies that $p^{2}$ divides $a + ib$.
    Thus, $p^2$ divides both $a$ and $b$, making it impossible for $(a,b,c)$ to
    be twist-minimal at $p$.
  \item Every prime factor of $c$ is $1 \md 4$, and $c > 0$.
  \item It follows from $a^2 + b^2 \equiv 1 \md 4$.
  \item If $a = 0$, then $a^2 = c^4$. This implies that $|a|$ is a
    square, and if $p \mid a$, then $p^2 \mid a$ and $p \mid c$. The
    $(2,2,1)$-minimality implies that $a$ has no prime divisors.
  \item Same as above.
  \item Follows from the above by the unique factorization of $\ZZ[i]$ and the
    multiplicativity of the norm $\Nm(\alpha) \colonequals \alpha\ol{\alpha}$.
  \end{enumerate}
\end{proof}

Define the \cdef{height zeta function} corresponding to \Cref{eq:twist-height}
to be the Dirichlet series
\begin{equation*}
  F(s) \colonequals \sum_{c = 1}^\infty \dfrac{f(c)}{c^s},
\end{equation*}
where $f(c)$ is the \cdef{arithmetic height function} that counts the number of
twist-minimal triples $(a',b',c')$ with $\twht(a',b',c') = c' = c$. The
following lemma describes the function $f(c)$ in terms of the prime
factorization of $c$. Recall that an arithmetic function $\phi \colon \ZZ_{>0}
\to \CC$ is \cdef{multiplicative} if $\phi(nm) = \phi(n)\phi(m)$ for every
relatively prime positive integers $n,m$. If moreover $\phi(nm)=\phi(n)\phi(m)$
for every $n,m \in \ZZ_{>0}$, then $\phi$ is called \cdef{completely
multiplicative}.


\begin{lemma}
  \label{lemma:arithmetic-height-fun-f}
    The arithmetic height function $f(c)$ satisfies the following properties.
    \begin{enumerate}
        \item If $p \not\equiv 1 \md 4$, then $f(p^k) = 0$ for every positive integer $k$.
        \item If $p \equiv 1 \md 4$, then $f(p^k) = 16$ for every positive integer $k$.
        \item The arithmetic function $f(c)/4$ is multiplicative, but not
          completely multiplicative.
    \end{enumerate}
\end{lemma}
\begin{proof}
    \hfill
    \begin{enumerate}
        \item Follows from \Cref{lemma:twist-min-triples}.
        \item For such a prime $p$, write $p\ZZ[i] = \pi\cdot\overline{\pi}$
          for the prime ideal factorization in the Gaussian ring. There are
          four ideals $I \subset \ZZ[i]$ that are not divisible by
          $\langle p \rangle^2 = \pi^2\cdot\overline{\pi}^2$ satisfying
          $N(I) \colonequals \#(\ZZ[i]/I) = p^{4k}$; they are:
        \begin{equation*}
          \pi^{4k}, \overline{\pi}^{4k}, \pi^{4k-1}\cdot\overline{\pi},
          \text{ and } \pi\cdot\overline{\pi}^{4k-1}.
        \end{equation*}
        Since there are four units in the principal ideal domain $\ZZ[i]$,
        we see that $f(c)/4$ counts the number of integral ideals $I \subset
        \ZZ[i]$ satisfying:
        \begin{itemize}
          \item $N(I) = c^4$, and
          \item $\langle p \rangle^2$ does not divide $I$ for any rational prime $p$.
        \end{itemize}
        \item This follows directly from the characterization of $f(c)/4$ given
          above, and the multiplicativity of the ideal norm.
    \end{enumerate}
\end{proof}

\begin{proof}[Proof of {\Cref{thm:FF-count}}]
    It follows from the multiplicativity of $f(c)/4$
    (\Cref{lemma:arithmetic-height-fun-f}), and the calculation of prime power
    values of $f$ (\Cref{lemma:twist-min-triples}), that the
    height zeta function $F(s)$ has the following Euler product expansion:
    \begin{align*}
      F(s) = 4\sum_{c=1}^\infty \tfrac14 f(c)c^{-s} =
      & \, 4\prod_{p}\paren{1 + \sum_{k=1}^\infty \tfrac14 f(p^k)p^{-ks} } \\
    = & \, 4\prod_{p \, \equiv\, 1 \md 4}\paren{1 + \sum_{k=1}^\infty 4p^{-ks}}
        = 4\prod_{p \, \equiv\, 1 \md 4}\paren{\dfrac{1+3p^{-s}}{1-p^{-s}}}.
    \end{align*}
    In particular, this calculation shows that $F(s)$ is absolutely convergent
    in the half-plane $\Re(s) > 1$. We can manipulate this Euler product to obtain
    \begin{equation*}
      F(s) = 4 \prod_{p \, \equiv\, 1 \md 4} \paren{1-p^{-s}}^{-4}
      \prod_{p \, \equiv\, 1 \md 4}\paren{1 -6p^{-2s}+8p^{-3s}-3p^{-4s}}.
    \end{equation*}
    At this stage, we introduce the Dedekind zeta function $\zeta_{\QQ(i)}(s)$
    of the Gaussian field $\QQ(i)$. If $\chi_4(n) = (-4 | n)$ is the
    non-trivial Dirichlet character of modulus $4$, we have the well-known
    identity $\zeta_{\QQ(i)}(s) = \zeta(s)L(s,\chi_4)$, where $\zeta(s)$ is
    the Riemann zeta function, and $L(s,\chi_4)$ is the Dirichlet $L$-function
    associated to $\chi_4(n)$.  We deduce the identity
    \begin{equation}
        \label{eq:F=xi.P}
        F(s) = \zeta(s)^2P(s),
    \end{equation}
    where
    \begin{equation*}
      P(s) \colonequals 4L(s,\chi_4)^2\paren{1-2^{-s}}^2
      \prod_{p \, \equiv\, 3 \md 4}\paren{1-p^{-2s}}^2
      \prod_{p \, \equiv\, 1 \md 4}\paren{1 -6p^{-2s}+8p^{-3s}-3p^{-4s}}.
    \end{equation*}
    From \Cref{eq:F=xi.P}, we see that $F(s)$ has a unique pole at $s = 1$
    of order $2$, and admits a meromorphic continuation to the half-plane
    $\Re(s) > 1/2$. Moreover, the residue of $F(s)$ at $s=1$ is
    \begin{align*}
      f_1 &= \Res_{s=1}(F(s)) = \Res_{s=1}(\zeta(s)^2)\cdot P(1) = 2\gamma_0P(1). \\
      P(1) &= 4\cdot(\pi/4)^2\cdot(1/4)\cdot \prod_{p \, \equiv\, 3 \md 4}\paren{1-p^{-2}}^2
      \prod_{p \, \equiv\, 1 \md 4}\paren{1 -6p^{-2}+8p^{-3}-3p^{-4}}.
    \end{align*}


    At this point, we can apply \Cref{thm:tauberian1} and obtain:
    \begin{equation*}
        N_{\FF}(T) = f_1 T(\log T) + f_2T + O\left(\dfrac{T}{\log T}\right).
      \end{equation*}
      To improve this error term, we can use \Cref{thm:tauberian2} with $k = 2$
      and $\delta = 1/2$ to conclude that for any $\epsilon > 0$,
      \begin{equation*}
        N_{\FF}(T) = f_1 T(\log T) + f_2T + O_\epsilon\left(T^{1/2 + \epsilon}\right).
      \end{equation*}
\end{proof}
    

\section{Counting minimal integer points on $x^2 + y^2 = z^4$}
\label{sec:counting-min-sols}

\begin{definition}
  \label{def:minimal-triple}
  We will say that a triple of integers $(a,b,c)$ is a
  \cdef{minimal triple} if the following conditions are satisfied:
  \begin{itemize}
   \item $a^2 + b^2 = c^4$,
   \item $(a,b,c)$ is $(4,4,2)$-minimal, and
   \item $c\neq0$.
  \end{itemize}
  The \cdef{minimality defect} of a triple $(a,b,c)$ is its $(4,4,2)$-greatest
  common divisor, and it is dfenoted by $\mind(a,b,c)$. By definition, every
  integral solution $(a,b,c)$ of $x^2 + y^2 = z^4$ is a $(4,4,2)$-twist of the
  twist-minimal solution $(a/d^4, b/d^4, c/d^2)$, where $d = \mind(a,b,c)$.
\end{definition}

The main result of this section is the asymptotic count of minimal triples. For
a real parameter $T > 0$, denote by $N_\GG(T)$ the number of twist minimal
triples with height less than or equal to $T$. We will prove the following
theorem.

\begin{theorem}
    \label{thm:GG-count}
    There exist explicitly computable constants $g_1, g_2, g_3 > 0$ such that for
    every $\epsilon > 0$, we have
    $$N_{\GG}(T) = g_1 T(\log T)^2 + g_2T(\log T) + g_3T + O_\epsilon(T^{1/2+\epsilon}),$$
    as $T \to \infty$. Furthermore, the constant $g_1$ is given by
    \begin{equation*}
      g_1 = \tfrac{9}{4}(\gamma_0^2+\gamma_1)\prod_{p \, \equiv\, 3 \md 4}\paren{1-p^{-2}}^2
      \prod_{p \, \equiv\, 1 \md 4}\paren{1 -6p^{-2}+8p^{-3}-3p^{-4}}.
      \end{equation*}
      where $\gamma_0$ is the Euler-Mascheroni constant.
\end{theorem}

As before, we aim to understand the analytic properties of the correspondig
\cdef{height zeta function}
\begin{equation*}
    G(s) \colonequals \sum_{n=1}^\infty \frac{g(n)}{n^{s}}, 
\end{equation*}
where $g(n)$ is the \cdef{arithmetic height function} that counts the number of
minimal triples $(a,b,c)$ with $\Ht(a,b,c) = |c| = n$.

Instead of directly analyzing the zeta function $G(s)$, we leverage our
understanding of $F(s)$, in the spirit of \cite[Theorem 5.1.4]{MV22}.

\begin{proof}[Proof of {\Cref{thm:GG-count}}]
    For a positive integer $n$, let 
    \begin{align*}
        \FF_n &\colonequals \brk{(a,b,c) : \text{twist-minimal, with } |c| = c = n}, \\
        \GG_n &\colonequals \brk{(a,b,c) : \text{minimal, with } |c| = n}
    \end{align*}
    be the level sets of points of height $n$, so that $f(n) = \#\FF_n$ and
    $g(n) = \#\GG_n$. For every square-free $d \in \ZZ$, let
    \begin{equation*}
        g^{(d)}(n) \colonequals \#\brk{(a,b,c) \in \ZZ^3 : (d^2a, d^2b, dc) \in \GG_n}.
    \end{equation*}
    It follows from the definitions that
    \begin{equation}
        g^{(d)}(n) = 
        \begin{cases}
            2f(n/|d|), & \mif d \mid n, \\
            0, & \mif d \nmid n.
        \end{cases}
    \end{equation}
    Therefore,
    \begin{align*}
        g(n) = \sum_{\substack{d \in \ZZ \\ \square\text{-free}}} g^{(d)}(n) = 4\sum_{\substack{d > 0 \\ d \mid n}} \mu^2(d)f(n/d) = 4(\mu^2*f)(n).
    \end{align*}
    It follows that
    $G(s) = 4\zeta(s)F(s)/\zeta(2s) = 4\zeta^3(s)P(s)/\zeta(2s)$, and applying
    the Tauberian theorem once more gives the result. In particular, we have
    that
    \begin{align*}
      g_1 &= \tfrac12 \Res_{s=1}(G(s)) \\
          &= 2\Res_{s=1}(\zeta(s)^3)\cdot P(1)\cdot\tfrac{6}{\pi^2}\\
          &= \tfrac{9}{4}(\gamma_0^2+\gamma_1)\prod_{p \, \equiv\, 3 \md 4}\paren{1-p^{-2}}^2
      \prod_{p \, \equiv\, 1 \md 4}\paren{1 -6p^{-2}+8p^{-3}-3p^{-4}}.
    \end{align*}
\end{proof}
    

\section{Elliptic curves and $5$-isogenies}
\label{sec:elliptic-curves-5isog}

We use the notation established in \cite{MV22}. Let $E$ be an elliptic curve
over $\QQ$. Then $E$ admits a short Weierstrass model $E_{A,B}\colon y^2 = x^3
+ Ax + B$ with $A,B \in \QQ$. Furthermore, two such models are isomorphic if
their coefficients satisfy
\begin{equation*}
  (A', B') = (u^4A, u^6B) \text{ for some } u \in \QQ\unit.
\end{equation*}
It follows that we can choose a Weierstrass pair $(A,B) \in \ZZ^2$ to represent
the $\QQ$-isomorphism class of $E$. The \cdef{height} of $(A,B)$ is
$H(A,B) \colonequals \max(4|A|^3, 27|B|^2)$. The \cdef{minimality defect} of
$(A,B)$, denoted $\mind(A,B)$, is the $(4,6)$-weighted common divisor of $A$
and $B$, i.e., largest $d \in \ZZ_{>0}$ such that $d^4 \mid A$ and
$d^6 \mid B$. Every elliptic curve $E$ defined over $\QQ$ with has a unique
short Weierstrass model over, up to $\QQ$-isomorphism, with minimality defect
$1$. We call this the \cdef{minimal model} of $E$, and it is given by
\begin{equation}
    \label{eq:minimal-model}
    y^2 = x^3 + (A/d^4)x + B/d^6, \quad \text{ for } d = \mind(A,B),
\end{equation}
where $E_{A,B}$ is any short Weierstrass model for $E$. We call
\Cref{eq:minimal-model} the \cdef{minimal model} of $E$.




The \cdef{twist minimality defect} of
$(A,B)$, denoted $\twmind(A,B)$, is the $(2,3)$-weighted common divisor of $A$
and $B$, i.e., the largest $e \in \ZZ_{>0}$ such that $e^2 \mid A$ and
$e^3 \mid B$.

For any elliptic curve $E$ over $\QQ$ we choose a short Weierstrass model
$E_{A,B}$ and define the \cdef{height} of (the isomorphism class of) $E$ to be
\begin{equation}
    \label{eq:height}
    \height(E) = \height(A,B) \colonequals H(A,B)/\mind(A,B)^{12}.
\end{equation}
Similarly, the \cdef{twist height} of (the $\Qbar$-isomorphism class of) $E$ is
\begin{equation}
    \label{eq:twist-height}
    \twht(E) = H(A,B)/\twmind(A,B)^{6}.
\end{equation}
For $j = 0, 1728$, we choose twist minimal models as follows:
\begin{itemize}
\item If $j(E) = 0$ ($A =0$), then we take $y^2 = x^3 + 1$ of \cdef{twist
    height} $27$.
\item If $j(E) = 1728$ ($B=0$), then we take $y^2 = x^3 + x$ of \cdef{twist
    height} $4$.
\end{itemize}
Every elliptic curve $E$ defined over $\QQ$ has a unique model over $\QQ$, up
to $\Qbar$-isomorphism, with twist minimality defect $1$ and $B > 0$. We call
this the \cdef{twist minimal model} of $E$, and it is given by
\begin{equation}
    \label{eq:twist-minimal-model}
    y^2 = x^3 + (A/e^2)x + |B|/e^3, \quad \text{ for } e = \twmind(A,B).
\end{equation}

The coarse space of $Y_0(5)$ is an affine open in $\Pone$, so the objects of
interest are parametrized by it's coordinate $t \neq 0, \infty$.
\begin{lemma} \label{lemma:X0(5)-parametrization} The set of isomorphism
  classes of elliptic curves $E$ over $\QQ$ that admit a $5$-isogeny (defined
  over $\QQ$) are precisely those of the form
  $E_5^{(e)}\colon y^2 = x^3 + e^2f(t)x + e^3g(t)$ for some $e \in \ZZ$ square
  free and $t\in \QQ\unit$, where
    \begin{align}
    \label{eq:f}
        f(t) &= -3(t^2 + 10t + 5)(t^2 + 22t + 125), \\
    \label{eq:g}
        g(t) &= 2(t^2 + 4t - 1)(t^2 + 22t + 125)^2.
    \end{align}
\end{lemma}
\begin{proof}
  Elliptic curves over $\QQ$ admitting a $5$-isogeny have $j$-invariant of the
  form (see \cite[Page 1247]{halberstadt21})
    \begin{equation}
        \label{eq:j}
        j = (t^2 + 10t + 5)^3 / t.
    \end{equation}
    From this we get a minimal Weierstrass equation
    $E_5\colon y^2 = x^3 + f(t)x + g(t)$ with $f(t), g(t) \in \ZZ[t]$ as in the
    assertion. A calculation shows that $\disc(E_5) = 12^6t(t^2 + 10t +5)^3$,
    so that the only values of $t\in \QQ$ for which $E_5$ specializes to a
    singular cubic curve is $t=0$. For future reference, let
    \begin{equation}
        \label{eq:h}
        h(t) \colonequals t^2 + 22t + 125.
    \end{equation}
    This polynomial is the greatest common divisor between $f(t)$ and $g(t)$.
    The $5$-division polynomial of the curve $E_2$ has an irreducible quadratic
    factor:
    \begin{equation}
        x^2 -2h(t)x +(t^2 + 22t + 89)h(t),
    \end{equation}
    verifying the claim that the curves $E_5^{(e)}(t)$ admit a five isogeny.
\end{proof}



\section{Minimal triples and $5$-isogenies}
\label{sec:triples-and-5isog}

Given a triple of integers $(a,b,c)$ satisfying the equation $x^2 +
y^2 = z^4$, we define a short Weierstrass equation
\begin{equation}
  \label{eq:Eabc}
  y^2 = x^3 + A(a,b,c)x + B(a,b,c),
\end{equation}
where
\begin{align}
  \label{eq:A}
  A(a,b,c) &\colonequals -6(123a + 114b + 125c^2) \\
  &= -(2\cdot 3^2\cdot 41)a - (2^2\cdot 3^2 \cdot 19)b - (2\cdot 3\cdot 5^3)c^2, \notag\\
  \label{eq:B}
  B(a,b,c) &\colonequals 8c(2502a + 261b + 2500c^2) \\
  &= (2^4\cdot 3^2\cdot 139)ac + (2^3\cdot 3^2\cdot 29)bc + (2^5\cdot 5^4)c^3. \notag
\end{align}
The discriminant of this equation is $\Delta(a,b,c) \colonequals
-16(27A(a,b,c)^3 + 4B(a,b,c)^2)$. When this quantity is not zero,
\Cref{eq:Eabc} defines an elliptic curve $E_{a,b,c}$ over $\QQ$.

\begin{theorem}
  \label{thm:Eabc}
  Suppose $(a,b,c)$ is an integer solution to $x^2 + y^2 = z^4$ that is not
  $(2,2,1)$-equivalent to $(-1,0,1)$ or $(-585,220,25)$. Then,
  \begin{enumerate}[label=(\alph*)]
      \item $\Delta(a,b,c)
  \neq 0$ and $E_{a,b,c}$ is an elliptic curve over $\QQ$
  admitting a $5$-isogeny $\phi_{a,b,c}$.
      \item Two triples are $(2,2,1)$-equivalent by $0 \neq d\in \ZZ$, say
  $(a_2,b_2,c_2) = (d^2a_1,d^2b_1,dc_1)$, if and only if the corresponding
  pairs $(E_1,\phi_1)$ and $(E_2,\phi_2)$ are $\QQ(\sqrt{d})$-twists.
      \item Every pair $(E,\phi) \in Y_0(5)(\QQ)$ is $\QQ$-isomorphic to
  $(E_{a,b,c},\phi_{a,b,c})$ for a unique minimal triple $(a,b,c)$.
      \item The set of quadratic twists of a pair $(E_{a,b,c},\phi_{a,b,c}) \in
        Y_0(5)(\QQ)$ corresponds to the set of $(2,2,1)$-twists of the minimal
        triple $(a,b,c)$.
  \end{enumerate}
\end{theorem}

\section{Counting $5$-isogenies up to $\Qbar$-twist}
\label{sec:counting-5-isogenies-twist}

We seek to leverage our count of twist-minimal triples $N_\FF(T)$ to count
$5$-isogenies of elliptic curves over $\QQ$ up to twist, since by
\Cref{thm:Eabc}, we have a bijective correspondence between these. The issue,
is that we wish to count $5$-isogenies by naive height, and this height is far
from convenient.

Let $(a,b,c)$ be a Fermat triple. Define
\begin{align*}
  H(a,b,c) &\colonequals H(A(a,b,c), B(a,b,c)) \\
           &= \max(4|A(a,b,c)|^3,27|B(a,b,c)|^2) \\
           &= 2^5\cdot 3^3\max(|123a + 114b + 125c^2|^3,2c^2|2502a + 261b + 2500c^2|^2). \\
\end{align*}
If $\phi_{a,b,c}\colon E_{a,b,c} \to E_{a,b,c}'$ is the $5$-isogeny
corresponding to a twist-minimal triple $(a,b,c)$, then its twist height is
not necesarily equal to $H(a,b,c)$. This is because the model chosen for
$E_{a,b,c}$ need not be twist-minimal.

\begin{definition}
  Define the \cdef{twist error} of a Fermat triple $(a,b,c)$, denoted by $\twerr(a,b,c)$, to be the twist
  minimality defect of the elliptic curve $E_{a,b,c}$. In other words,
  \begin{equation*}
    \twerr(a,b,c) \colonequals \twmind(A(a,b,c),B(a,b,c)).
  \end{equation*}
\end{definition}

This is the quantity needed to calculate the twist height of $E_{a,b,c}$.
\begin{equation}
  \label{eq:twHt-Eabc}
  \twht(E_{a,b,c}) = H(a,b,c)/\twerr(a,b,c)^6.
\end{equation}
Our first task is to understand the quantity $\twerr(a,b,c)$. We break this up
into a number of lemmas.

\begin{lemma}
  \label{lemma:p-mid-c-and-pp-mid-A}
  Let $(a,b,c)$ be a twist-minimal triple. If a prime $p$ satisfies
  \begin{equation*}
    p \mid c,\quad \mand \quad p^2 \mid A(a,b,c),
  \end{equation*}
  then $p=5$.
\end{lemma}
\begin{proof}
  We know from \Cref{lemma:twist-min-triples} that every prime $p$ dividing $c$
  satisfies $p \equiv 1 \md 4$, so it is not possible to have $p = 2,3$
  dividing $c$. Thus, we already know that $p \neq 2,3$.

  Assume that $p \neq 5$. It follows from reducing \Cref{eq:A} modulo $p^2$
  that $123a \equiv -114b \md p^2$. This implies that
  \begin{align*}
    (123c^2)^2 &= (123a)^2 + (123b)^2 \\
               &\equiv (-114b)^2 + (123b)^2 \md p^2 \\
               &\equiv (114^2 + 123^2)b^2 \md p^2 \\
               &\equiv 3^2\cdot 5^5\cdot b^2 \equiv 0 \md p^2 .
  \end{align*}

It follows that $p \mid b$, and therefore $p \mid a$ as well. Let $a'
\colonequals a/p$, $b' \colonequals b/p$ and $c'\colonequals c/p$. We have that
$(a')^2 + (b')^2 = p^2(c')^4$ and $123a' + 114b' + 125p(c')^2 \equiv 0 \md p$.
Once more, it follows that $123a' \equiv -114b' \md p$, and
  \begin{align*}
    (123a')^2 + (123b')^2 &\equiv (114^2 + 123^2)(b')^2 \md p \\
                       &\equiv 3^2\cdot 5^5\cdot (b')^2 \equiv 0 \md p .
  \end{align*}
  This allows us to conclude that $p \mid b'$, and therefore $p \mid a'$ as well. This contradicts the
  twist-minimality of the triple $(a,b,c)$.

  To see that the case $p=5$ indeed occurs, one can take $(a,b,c) = (-527,
  -336, 25)$. This triple is twist-minimal, since $\gcd(a,b)=1$, and $A(a,b,c)
  = 2^4\cdot 3 \cdot 5^5$.
\end{proof}

\begin{lemma}
  Let $(a,b,c)$ be a twist-minimal triple. If $p$ is a prime divisor of
  $\twerr(a,b,c)$, then $p \in \brk{2,5}$.
\end{lemma}
\begin{proof}
  By definition, $p \mid \twerr(a,b,c)$ if and only if
  $p^2 \mid A \colonequals A(a,b,c)$ and $p^3 \mid B \colonequals B(a,b,c)$. If we had that $p$ also
  divides $c$, then \Cref{lemma:p-mid-c-and-pp-mid-A} implies that
  $p = 5$. For this reason, we assume that $p$ does not divide $c$.

  We use the equations $A \equiv 0 \md p^2$ and $B \equiv 0 \md p^2$ to find
  congruence relations between $a$ and $b$ modulo $p^2$. We find an integer
  linear combinations of $cA$ and $B$ that allow us to cancel the terms with
  $a$ or $b$ in them, obtaining:
  \begin{align*}
    1112cA + 41B &= -(14000c^3 + 675000bc) = -2^35^3c(14c^2 + 675b) \equiv 0 \md
    p^2, \\
    58ca + 19B &= 337500ac + 336500c^3 = 2^25^3c(673c^2+675a) \equiv 0 \md p^2.
  \end{align*}
  If $p \neq \brk{2,5}$, this implies that $(14c^2 + 675b) \equiv 0 \md
    p^2$ and $(673c^2+675a) \equiv 0 \md p^2$. But in this case, we can reduce
    the equation $(675a)^2 + (675b)^2 = 675^2c^4$ modulo $p^2$ to obtain
    \begin{equation*}
      5^6\cdot 29 = 14^2 + 673^2 \equiv 675^2 = 3^6\cdot 5^4  \md p^2,
    \end{equation*}
    producing a contradiction. We conclude that in this case $p \in \brk{2,5}$,
    and the result follows.
\end{proof}

\begin{lemma}
  \label{lemma:exceptional-triples}
  Let $(a,b,c)$ be a twist-minimal triple.
  \begin{enumerate}[label=(\alph*)]
  \item \label{item:2-exceptional} $2 \mid \twerr(a,b,c)$ if and only if
    $2 \mid b$.
  \item \label{item:5-exceptional} $5 \mid \twerr(a,b,c)$ if and only if
    $a \equiv 7b \md 25$ and $5 \mid c$.
  \end{enumerate}
  We say that $(a,b,c)$ is \cdef{2-exceptional} when \ref{item:2-exceptional}
  holds, and \cdef{5-exceptional} when \ref{item:5-exceptional} holds.
\end{lemma}
\begin{proof}
  \hfill
  \begin{enumerate}[label=(\alph*)]
  \item If $2 \mid \twerr(a,b,c)$, then $4 \mid A$. This implies that $a$ and
    $c$ have the same parity. Since $c \equiv 1 \md 4$, $a$ must be odd. But
    since $a$ and $b$ have oposite parity, we conclude that $2 \mid b$.
    Conversely, suppose that $2 \mid b$. This implies that $a$ is odd. Since $9
    \mid B$ always, we only need to show that $4 \mid A$. But this is visibly
    true once we know that $a$ is odd.
  \item If $5 \mid \twerr(a,b,c)$, then $A \equiv 0 \md 25$. This implies that
    $a \equiv 7b \md 25$. From this congruence, we see that $c^4 = a^2 +
    b^2 = 50b^2 \equiv 0 \md 25$, so that $5 \mid c$. Conversely, the
    congruence $a \equiv 7b \md 25$ implies that $A \equiv 0 \md 25$ and $B/(8c)
    \equiv 0 \md 25$, which is enough to conclude that $5 \mid \twerr(a,b,c)$.
  \end{enumerate}
\end{proof}

\begin{lemma}
  \label{lemma:4-nmid-twerr}
  Let $(a,b,c)$ be a twist-minimal triple. Then
  $\twerr(a,b,c)$ is not divisible by $4$.
\end{lemma}
\begin{proof}
  Suppose that $4 \mid \twerr(a,b,c)$. By definition, this means that $4^2 \mid
  A$ and $4^3 \mid B$. From \Cref{eq:A,eq:B}, we deduce that
  \begin{align*}
    123a + 114b + 125c^2 &\equiv 0 \md 8, \\
    2502a + 261b + 2500c^2 &\equiv 0 \md 8.
  \end{align*}
  Since $c \equiv 1 \md 4$ (\Cref{lemma:twist-min-triples}), we have that
  $c^2 \equiv 1 \md 8$, and the above congruences simplify to:
    \begin{align*}
    3a + 2b + 5 &\equiv 0 \md 8, \\
    6a + 5b + 4 &\equiv 0 \md 8.
    \end{align*}
    These imply that $b \equiv 6 \md 8$, and $a^2 + b^2 = c^4$ imlpies that
    $a^2 \equiv 5 \md 8$. But this contradicts the fact that $5$ is not a
    square modulo $8$.
\end{proof}

\begin{lemma} \label{lemma:625-nmid-twerr}
    Let (a,b,c) be a twist-minimal triple. Then $\twerr(a,b,c)$ is not divisible by $5^4$.
\end{lemma}
\begin{proof}
    Suppose that $5^4 \mid \twerr(a,b,c)$. By definition, this implies that $5^8 \mid A$ and $5^{12} \mid B$. Using \Cref{eq:A,eq:B}, we deduce that
    \begin{align*}
        123a + 114b + 125c^2 &\equiv 0 \text{ mod } 5^8 \\
        c(2502a + 261b + 2500c^2) &\equiv 0 \text{ mod } 5^{12}.
    \end{align*}
    We consider 3 cases depending on the valuation of $c$ with respect to $5$.

    \medskip \textbf{Case 1}: Suppose $5^3 \mid c$. Then,
    $41 a + 38b \equiv 0 \md 5^8$. This congruence simplifies to
    $a \equiv 323932b \md 5^8$. Using the equation $a^2 + b^2 = c^4$, we obtain
    \begin{equation*}
        a^2 + b^2 \equiv 300000a^2 = 2^5 \cdot 3 \cdot 5^5 a^2 \equiv 0 \md 5^8.
    \end{equation*}
    This implies that $a \equiv b \equiv 0 \text{ mod } 5^2$, contradicting
    the twist-minimality of $(a,b,c)$.

    \medskip 
    \textbf{Case 2}: Suppose that $5^2 \mid c$ but $5^3 \nmid c$. We obtain the
    linear system
    \begin{align*}
        41a + 38b &\equiv 0 \md 5^7, \\
        278a + 29b &\equiv 0 \md 5^7.
    \end{align*}
    Solving for $b$ in the firs equation and substituting into the second one,
    we arrive at
    \begin{equation*}
        18750a = 2 \cdot 3 \cdot 5^5 a \equiv 0 \md 5^7.
    \end{equation*}
    This implies that $a \equiv b \equiv 0 \md 5^2$, contradicting
    the twist-minimality of $(a,b,c)$.

    \medskip \textbf{Case 3}: Suppose that $5 \mid c$ but $5^2 \nmid c$. Then
    $41a + 38b \equiv 0 \md 5^5$. Solving for $b$ we obtain
    $b \equiv 1068a \md5^5$, which implies
    \begin{equation*}
        a^2 + b^2 \equiv 3125 a^2 \equiv 5^5 a^2 \equiv 0 \equiv c^4 \text{ mod } 5^5.
    \end{equation*}
    This implies that $5^2 \mid c$, a contradiction.

    \medskip \textbf{Case 4}: Suppose that $5 \nmid c$. Then \Cref{eq:B}
    implies $278a + 29b \equiv 0 \text{ mod } 5^4$. Solving for $b$, we obtain
    $b \equiv 497 \md 5^4$, which implies
    \begin{equation*}
       c^4 = a^2 + b^2 \equiv 625 a^2 \equiv 5^4 a^2 \equiv 0 \md5^4.
    \end{equation*}
    This implies that $5 \mid c$, a contradiction.
\end{proof}


\spark{For example, suppose that $d=1$. Then Lemma 7.7 shows that if $c = 5$,
  then there are $8$ twist-minimal triples such that $5 \mid \twerr(a,b,c)$ and
  $25 \nmid \twerr(a,b,c)$. These are:
\begin{equation*}
    a+bi \in \{-15-20i, 20-15i, 15 + 20i, -20+15i, -7 + 24i, -24-7i, 7 - 24i, 24 + 7i\}
\end{equation*}
These 8 elements correspond to the following triples:
\begin{equation*}
    (a,b) \in \{(-15,-20),(20,-15),(15,20),(-20,15),(-7,24),(-24,-7),(7,-24),(24,7)\}
\end{equation*}
This agrees with the result of the updated magma code commented below.
}

% ZZ := Integers();
% t := PolynomialRing(Integers()).1;
% K<i> := NumberField(t^2 + 1);

% candidates := [];
% for k := 1 to 10 do
%     for pi in {1 + 2*i, 1-2*i} do;
%         c := 5^k;
%         for n :=0 to 3 do
%             alpha := i^n*pi^(4*k);
%             a := ZZ!(alpha[1]);
%             b := ZZ!(alpha[2]);
%             if IsDivisibleBy(7*a + b,5^2) then Append(~candidates, [a,b,c]); end if;
%             beta := i^n*pi^(4*k-2);
%             a := 5*ZZ!(beta[1]);
%             b := 5*ZZ!(beta[2]);
%             if IsDivisibleBy(7*a + b,5^2) then Append(~candidates, [a,b,c]); end if;
%         end for;
%     end for;
% end for;
% print(candidates);

\begin{lemma}
  Let $(a,b,c)$ be a $5$-exceptional triple corresponding to
  $\alpha = a+bi \in \ZZ[i]$. Then, we can write $\alpha = \beta\cdot\delta$
  where $\beta$ and $\delta$ are coprime twist-minimal Gaussian integers, and
  $\beta$ is of the form described in
  \Cref{table:characterization-5-exceptional-triples}.
\end{lemma}
\begin{table}[ht]
    \begin{tabular}{|c|c|}
      \hline
      $\ord_5(\twerr(a,b,c))$ & $\beta$               \\ \hline
      $1$   & $(1 - 2i)^4$                            \\ \hline
      $1$   & $5(1 - 2i)^2$                           \\ \hline
      $2$   & $(1 - 2i)^{4k}$                         \\ \hline
      $3$   & $5(1 - 2i)^{4k-2}$                      \\ \hline
    \end{tabular}
    \caption{Characterization of $5$-exceptional triples. ($k \geq 2$). }
    \label{table:characterization-5-exceptional-triples}
\end{table}
\begin{proof}
  Since $(a,b,c)$ is $5$-exceptional, we know that $\ord_5(c) = k \geq 1$.
  Moreover, we can write $\alpha = \beta\cdot \delta$ so that $\delta$ is
  coprime to both $1\pm 2i$. This implies that $\beta \in \brk{(1\pm 2i)^{4k},
    5(1\pm 2i)^{4k-2}}$. On the other hand, observe that
  \begin{align}
   \label{eq:(1+2i)^5}
    (1+2i)^5  \alpha &= (41-38i)(a+bi) = (41a + 38b) + i(41b - 38a), \\
    \label{eq:(1+2i)^7}
        (1+2i)^7  \alpha &= (29+278i)(a+bi) = (29a -278b) + i(278a + 29b).
  \end{align}
  From these equations we conclude that\footnote{This can be seen by taking the
    norm and comparing the parity of $\ord_5(\cdot)$ on each side of the
    equations.}
  \begin{align*}
    \ord_5(41a + 38b) &= \ord_5(41b - 38a) = \ord_5((1+2i)^5\alpha), \\
    \ord_5(278a + 29b) &= \ord_5(29a -278b) = \ord_5((1+2i)^7\alpha).
  \end{align*}
  Furthermore, since $5$ divides $A$ and $B$, we must have that
  $\beta \in \brk{(1- 2i)^{4k}, 5(1-2i)^{4k-2}}$. Taking $\ord_5(\cdot)$ on
  \Cref{eq:A,eq:B} we obtain
  \begin{align}
    \label{eq:ord-5-A}
    \ord_5(A(a,b,c)) &\geq \min\brk{\ord_5((1+2i)^5\beta), 3+2k}, \\
    \label{eq:ord-5-B}
    \ord_5(B(a,b,c)) &\geq k + \min\brk{\ord_5((1+2i)^7\beta), 4+2k},
  \end{align}
  with equality when the entries of the minimum functions are distinct. From
  this, we are able to calculate the $5$-adic valuation of the twist error of
  the triple $\Fermat(\alpha) = (a,b,c)$. The values are sumarized in
  \Cref{table:beta-ord-5}.

  \begin{table}[ht]
    \begin{tabular}{|c|c|c|c|}
      \hline
      $\beta$          & $\ord_5(A)$ & $\ord_5(B)$ & $\ord_5(\twerr(a,b,c))$ \\\hline
      $(1-2i)^4$       & 4           & 5           & 1                       \\ \hline
      $5(1-2i)^2$      & 3           & 4           & 1                       \\ \hline
      $(1-2i)^{4k}$    & 5           & $k+7$       & 2                       \\ \hline
      $5(1-2i)^{4k-2}$ & 6           & $k+7$       & 3                       \\ \hline
    \end{tabular}
    \caption{Possible $\beta$ factors of $5$-exceptional triples.}
  \label{table:beta-ord-5}
  \end{table}
\end{proof}
  
\spark{Here's a question we would have to figure out. Based on the two previous
  lemmas, the possible values of $\twerr(a,b,c)$ are:
  $1, 2, 5, 10, 25, 50, 125, 250$. We would need to verify whether the
  twist-minimal triples with such values of $\twerr(a,b,c)$ give rise to
  non-isomorphic families of elliptic curves, up to quadratic twists over
  $\mathbb{Q}$.} \santi{I am not sure I understan your comment. Let me know if
  what I wrote after this possibly useless lemma is related to it.}
\begin{lemma}
  Let $(a,b,c)$ be a Fermat triple. Then,
  \begin{equation}
    \label{eq:ineq}
    H(a,b,c) \leq 2^6\cdot 3^3\cdot 5^8\cdot 13^2 \cdot c^6.
  \end{equation}
\end{lemma}
\begin{proof}
  From the triangle inequality $\Re(z), \Im(z) \leq \sqrt{z\ol{z}}$ applied to
  \Cref{eq:(1+2i)^5,eq:(1+2i)^7}, we obtain
  \begin{align*}
    |123a+114b+125c^2| &\leq 4\cdot 5^3\cdot c^2, \\
    |2502a+261b+2500c^2| &\leq 13\cdot 5^4\cdot c^2.
  \end{align*}
  Thus, $H(a,b,c)$ is upper bounded by
  $2^5\cdot 3^3(2c^2)(13\cdot 5^4\cdot c^2)^2$.
\end{proof}

From \Cref{thm:Eabc} that a twist-minimal Gaussian integer $\alpha = a+bi$
different from $-1$ and $-585+220i$ corresponds to a $\Qbar$-isomorphism class of a
rational point $(E,\phi) \in Y_0(5)(\QQ)$, which is represented by the pair
$(E_{a,b,c},\phi_{a,c,b})$ defined in \Cref{sec:triples-and-5isog}. It follows
that $N_5^{\tw}(T)$ counts the number of twist-minimal Gaussian integers
$\alpha \in \ZZ[i]-\brk{-1,585+220i}$ such that $\twht(E_\alpha) \leq T$. We
can partition this count according to the value of $\twerr(\alpha) \in
\brk{1,2,5,10,25,50,125,250}$ to obtain
\begin{equation*}
  N_5^{\tw}(T) = \sum_{r =0}^3[N(T; \twerr = 5^r) + N(T; \twerr
  = 2\cdot 5^r)],
\end{equation*}
where $N(T;\twerr=n)$ counts the number of twist-minimal Gaussian integers
$\alpha \in \ZZ[i]-\brk{-1,585+220i}$ such that $\twerr(\alpha) = n$ and 
$\twht(E_\alpha) = \max(4|A(\alpha)|^3,27|B(\alpha)|^2)/n^6 \leq T$. 

Suppose $\mathfrak{d} \in \{1, 2, 5, 10, 25, 50, 125, 250\}$. Recall that these are all possible values of twist errors of a Fermat triple $(a,b,c)$. Denote by $\FF(T; \twerr = \mathfrak{d})$ the following set
\begin{equation*}
    \FF(T; \twerr = \mathfrak{d}) := \{(a,b,c) \in \mathbb{Z}^3 \; | \; (a,b,c) \text{ twist minimal}, \twerr(a,b,c) = \mathfrak{d}, |c| \leq T\}.
\end{equation*}
We denote by $f_{\mathfrak{d}}(c)$ the twist error arithmetic height function function
\begin{equation*}
    f_{\mathfrak{d}}(c) := \# \{(a,b) \in \mathbb{Z}^2 \; | \; (a,b,c) \in \FF(T; \twerr = \mathfrak{d})\}.
\end{equation*}
We use the notation $N_{\FF}(T; \twerr=\mathfrak{d})$ to denote the number of elements in $\FF(T, \twerr = \mathfrak{d})$. Note that
\begin{equation*}
    N_{\FF}(T; \twerr = \mathfrak{d}) = \sum_{c = 1}^T f_{\mathfrak{d}}(c).
\end{equation*}

We first understand explicit expressions for $f_{\mathfrak{d}}(c)$ for all possible twist errors $\mathfrak{d}$.
\begin{lemma}
    The arithmetic height function $f_{\mathfrak{d}}(c)$ satisfies the following properties.
    \begin{enumerate}
        \item If $p \not\equiv 1 \text{ mod } 4$, then $f_{\mathfrak{d}}(p^k) = 0$ for every positive integer $k$.
        \item If $p \equiv 1 \text{ mod } 4$ and $p \neq 5$, then $f_{\mathfrak{d}}(p^k) = 8$ for every positive integer $k$.
        \item Suppose $p = 5$. 
        \begin{enumerate}
            \item If $\mathfrak{d} \in \{1, 2\}$, then $f_{\mathfrak{d}}(5^k) = 4$ for every positive integer $k$.
            \item If $\mathfrak{d} \in \{5,10\}$, then $f_{\mathfrak{d}}(5) = 4$, and $f_{\mathfrak{d}}(5^k) = 0$ for every positive integer $k \geq 2$.
            \item If $\mathfrak{d} \in \{25,50,125,250\}$, then $f_{\mathfrak{d}}(5) = 0$, and $f_{\mathfrak{d}}(5^k) = 2$ for every positive integer $k \geq 2$.
        \end{enumerate}
        \item The arithmetic function $f_{\mathfrak{d}}(c)/2$ is multiplicative, but not completely multiplicative.
    \end{enumerate}
\end{lemma}
\begin{proof}
    To be filled in.
\end{proof}

To relate point count estimates between $N(T; \twerr = \mathfrak{d})$ and $N_{\FF}(T; \twerr = \mathfrak{d})$, we use the angular equidistribution of Gaussian integers, as proposed from the work of Erd\"os and Hall \cite{EH99}. We first introduce the conditions imposed over compact regions of $\mathbb{R}^2$ over which Erd\"os and Hall's result can be effectively employed.
\begin{definition} \label{def:Property*}
    Let $C \subset \mathbb{R}^2$ be any compact region. We say that $C$ satisfies Property (*) if the following two conditions are satisfied.
    \begin{enumerate}
        \item There exist $T$ and a fixed constant $M > 0$ such that $B_{T^2} \subset C \subset B_{(M T)^2}$.
        \item The boundary of $C$, denoted as $\partial C \subset \mathbb{R}^2$, is defined as a continuous, piecewise differentiable, positive function $r = F(\theta)$, where $r, \theta$ are polar coordinates of $\mathbb{R}^2$.
    \end{enumerate}
    Given a compact region $C$ satisfying Property (*), we denote by $\varphi_C(x): [1,M] \to [0,1]$ the function defined as
    \begin{equation}
        \varphi_C(x) := 1 - \frac{1}{2\pi} \mu \left( \left\{ \theta \in [0,2\pi] \; : \; x \geq \frac{\sqrt{F(\theta)}}{T}\right\} \right).
    \end{equation}
    where $\mu(\cdot)$ is the canonical Lebesgue measure on $[0,2\pi]$.
\end{definition}
Note that $\varphi_C$ is a continuous piecewise differentiable decreasing function over $[1,M]$. In particular, $\varphi'_C(x) < 0$ for all $x \in [1,M]$.

Provided below is an adaptation of Erd\"os and Hall's work that fits to the setup of our problem.
\begin{theorem} \label{theorem:equidistribution}
    Denote by $B_r \subset \mathbb{R}^2$ the ball centered at the origin of radius $r$. Let $C \subset \mathbb{R}^2$ be a fixed compact region satisfying Property (*) from Definition \ref{def:Property*}.
    
    Suppose for any $T \geq 1$, for each $\mathfrak{d} \in \{1, 2, 5, 10, 25, 50, 125, 250\}$, and any small enough $\epsilon > 0$, there exists a fixed constant $0 < a < 1$ such that
    \begin{equation*}
        N_{\FF}(T; \twerr = \mathfrak{d}) = f_{1,\mathfrak{d}} T \log T + f_{2,\mathfrak{d}} T + O_\epsilon(T^{a + \epsilon}).
    \end{equation*}
    Then for sufficiently large $T$ and any small enough $\epsilon > 0$,
    \begin{equation*}
        \#(C \cap \FF(T; \twerr = \mathfrak{d})) =  f_{1,\mathfrak{d}} \cdot \left( -\int_1^M z \varphi'_C(z)dz \right) \cdot T \log T + O_\epsilon(T (\log T)^{1 - \kappa + \epsilon})
    \end{equation*}
    where $\kappa = -\frac{1}{2} \log(\frac{4 \sin(\pi/14) + 6\sin(\pi/7) - 8 \sin(3\pi/14) + 3 \cos(\pi/14) + 8 \cos(\pi/7) - 6 \cos(3\pi/14)}{3\pi}) \sim 0.4351914736$.
\end{theorem}
\begin{proof}

\medskip 
\textbf{Setup}
\medskip

    Given an integer $m$, denote by $A_m := B_{m + \frac{1}{2}} \setminus B_{m - \frac{1}{2}}$ the annulus of width $1$ with radius $m$ centered at the origin.
    It suffices to prove that for any positive integer $T^2 < m < MT^2$, the elements in the set 
    \begin{equation*}
        A_m \cap C \cap \FF(T; \twerr = \mathfrak{d})
    \end{equation*}
    is equidistributed inside the set 
    \begin{equation*}
        A_m \cap C.
    \end{equation*}
    Let $\mathbb{G}$ be the set of integers, all of whose prime factors are equivalent to $1$ modulo $4$. Denote by $\mathbb{G}(x)$ the subset of integers $c \in \mathbb{G}$ such that $|c| \leq x$. Given a twist minimal triple $(a,b,n) \in \FF(T; \twerr = \mathfrak{d})$, we denote by $\phi_{s,\mathfrak{d}}$ the angle associated to the triple $(a,b,n)$. We use the following quantity $\Delta(c)$ to quantify the deviation of the angular distribution of Gaussian integers from the desired angular equidistribution: 
    \begin{equation}
        \Delta_\mathfrak{d}(c) := \text{max} \left\{ \# \{s : \phi_{s,\mathfrak{d}} \in [\alpha, \beta] \text{ mod } 2\pi\} - \frac{\beta - \alpha}{2\pi} f_{\mathfrak{d}}(c) \; | \; 0 \leq \alpha \leq \beta \leq 2 \pi \right\}
    \end{equation}
    Then we want to show that for a fixed $\epsilon > 0$, for almost all integers in $\mathbb{G}$, i.e. with at most $o(\frac{x}{\sqrt{\log x}})$ exceptions not exceeding $x$, we want to prove that
    \begin{equation} \label{eq:equid_0}
        \Delta_\mathfrak{d}(c) < f_{\mathfrak{d}}(c) \cdot (\log x)^{-\kappa + \epsilon}
    \end{equation}
    for the given value $\kappa$ outlined in the statement of the theorem. 
    
    For now, let us suppose that equation (\ref{eq:equid_0}) holds. For each $m$, we denote by $\alpha_m, \beta_m$ the angular interval which satisfies
    \begin{align}
        \begin{split}
            \alpha_m &:= \min\left[ \theta \in [0,2\pi] \; | \; (m^2 \cos \theta, m^2 \sin \theta) \in A_m \cap C \cap \FF(T; \twerr = \mathfrak{d}) \right] \\
            \beta_m &:= \max\left[ \theta \in [0,2\pi] \; | \; (m^2 \cos \theta, m^2 \sin \theta) \in A_m \cap C \cap \FF(T; \twerr = \mathfrak{d}) \right] 
        \end{split}
    \end{align} 
    Using the notations above, we obtain that
    \begin{align} \label{eq:sum_main}
    \begin{split}
        \#(C \cap \FF(T; \twerr = \mathfrak{d})) &= \sum_{0 \leq m \leq M R} \# A_m \cap C \cap \FF(T; \twerr = \mathfrak{d}) \\
        &= \sum_{0 \leq m \leq M R} \left[\frac{\beta_m - \alpha_m}{2\pi} \cdot f_\mathfrak{d}(m) + O(f_\mathfrak{d}(m) \cdot (\log R)^{-\kappa + \epsilon}) \right].
    \end{split}
    \end{align}
    Let $\varphi: [0,MT] \to [0,1]$ be a continuous function defined as
    \begin{equation*}
        \varphi(x) := 1 - \frac{1}{2\pi} \mu \left(\{ \theta \in [0,2\pi] : F(\theta) \leq x^2 \} \right)
    \end{equation*}
    where $\mu(\cdot)$ is the canonical Lebesgue measure on $[0,2\pi]$. Note that $\varphi$ is a decreasing function over $[T,MT]$ and $\varphi(z) = 1$ over $z \in [0,T]$.
    Then one can rewrite the summation from (\ref{eq:sum_main}) using Abel summation formula as
    \begin{align}
        \begin{split}
            (\ref{eq:sum_main}) &= \sum_{0 \leq m \leq MT} \varphi(m) \cdot f_\mathfrak{d}(m) \cdot (1 + O((\log T)^{- \kappa + \epsilon})) \\
            &= - \int_{T}^{MT} N_{\FF}(y; \twerr = \mathfrak{d}) \cdot \varphi'(y) \cdot (1 + O((\log T)^{- \kappa + \epsilon})) dy \\
            &= -\int_T^{MT} f_{1,\mathfrak{d}} \cdot y \cdot \log y \cdot \varphi'(y) (1 + O((\log T)^{- \kappa + \epsilon}) dy
        \end{split}
    \end{align}
    Apply change of variables $y = Tz$ to obtain
    \begin{align}
        \begin{split}
            (\ref{eq:sum_main}) &= - f_{1,\mathfrak{d}} T \log T \cdot \int_{1}^{M} z \cdot \varphi'(Tz) \cdot T dz + O(T (\log T)^{1-\kappa + \epsilon})
        \end{split}
    \end{align}
    By definition, we have $\varphi(Tz) = \varphi_{C}(z)$, implying that $\varphi'(Tz) \cdot T = \varphi_{C}'(z)$. Hence
    \begin{align}
        \begin{split}
            (\ref{eq:sum_main}) &= f_{1,\mathfrak{d}} \left( - \int_{1}^{M} z \varphi'_{C}(z) dz \right) \cdot T \log T + O(T (\log T)^{1-\kappa + \epsilon}).
        \end{split}
    \end{align}

\medskip
\textbf{Angular equidistribution}
\medskip
    
    The proof of the angular equidistribution result is the adaptation of the techniques of the proof outlined in \cite{EH99}. We first recall the following theorem from \cite{ET48}.
    \begin{theorem}[\cite{ET48}] \label{theorem:ET}
        If $\varphi_1, \varphi_2, \cdots, \varphi_n$ are real numbers and for every $1 \leq k \leq m$, 
        \begin{equation}
            |\sum_{v=1}^n e^{k i \varphi_v} | \leq \psi(k)
        \end{equation}
        for some function $\psi$, then for arbitrary $0 \leq \alpha \leq \beta \leq 2 \pi$, there exists a constant $C > 0$ such that
        \begin{equation}
            \left| \sum_{\alpha \leq \varphi_v \leq \beta \text{ mod } 2 \pi} 1 -  \frac{\beta - \alpha}{2 \pi} n \right| < C \cdot \left( \frac{n}{m+1} + \sum_{k=1}^m \frac{\psi(k)}{k} \right)
        \end{equation}
    \end{theorem}
    For each $t$, denote by $Z_t(n)$ to be the following sum
    \begin{equation}
        Z_t(n) := \sum_{s=1}^{f(n)} e^{ti\phi_s}
    \end{equation}
    where $\phi_s$ is the angle associated to the solution $(a_s, b_s)$ such that $(a,b,n)$ is a twist minimal triple. (Note that some of the angles $\phi_{s}$ are equal to $\phi_{s',\mathfrak{d}}$ for some index $s'$ and twist minimal error $\mathfrak{d}$.) We recall that $f(n)$ is the arithmetic height function introduced in Lemma \ref{lemma:arithmetic-height-fun-f}.

    Using Theorem \ref{theorem:ET}, we obtain that uniformly in $T$,
    \begin{equation}
        \left| \sum_{\alpha \leq \phi_{s,\mathfrak{d}} \leq \beta \text{ mod } 2\pi} 1 - \frac{\beta - \alpha}{2\pi} f_\mathfrak{d}(n) \right| \ll \frac{f_\mathfrak{d}(n)}{T} + \sum_{t \leq T} \frac{|\sum_{s=1}^{f_{\mathfrak{d}}(n)} e^{t i \phi_{s,\mathfrak{d}}}|}{t} < \frac{f(n)}{T} + \sum_{t \leq T} \frac{|Z_t(n)|}{t}
    \end{equation}
    because by definition we have $f_\mathfrak{d}(n) \leq f(n)$ for every $n \in \mathbb{N}$.
    
    Restricted to $\mathbb{G}$, $\frac{Z_t(n)}{4}$ is a multiplicative function, as pointed out in Equation (17) of \cite{EH99}. In particular, for each prime $p \in \mathbb{G}$, one has
    \begin{equation}
        \frac{Z_t(p^k)}{4} = 1 + 2 \cos(4kt \theta(p)) + 2 \cos((4k-1)t \theta(p))
    \end{equation}
    where $\theta(p)$ is the argument of the Gaussian prime $w$ in the first octant such that $w \cdot \overline{w} = p$.

    Proceeding as in Equations (18) - (21) of \cite{EH99}, we have
    \begin{equation} \label{eq:thm:equid_1}
        \sum_{n \in \mathbb{G}(x)} \frac{\Delta_\mathfrak{d}(n)}{f_\mathfrak{d}(n)} y^{\Omega(n)} \ll \frac{1}{T} x (\log x)^{\frac{y}{2} - 1} + \sum_{t \leq T} \frac{1}{t} \sum_{n \in \mathbb{G}(x)} \frac{|Z_t(n)|}{f_\mathfrak{d}(n)} y^{\Omega(n)}
    \end{equation}
    where $\Omega(n)$ is the total number of prime factors of $n$. Using the fact that $f_\mathfrak{d}(n)/2$ is a multiplicative function, the fraction $\frac{|Z_t(n)|}{2 f_\mathfrak{d}(n)}$ is a multiplicative function. Use the fact that $f_{\mathfrak{d}}(n) \geq 2$ for all $n \in \mathbb{G}$ to we obtain
    \begin{align*}
    \begin{split}
        (\ref{eq:thm:equid_1}) &\ll \frac{1}{T} x (\log x)^{\frac{y}{2} - 1} + \frac{1}{2} \cdot \sum_{t \leq T} \frac{x}{t \log x} \cdot \prod_{\substack{p \leq x \\ p \equiv 1 \text{ mod } 4}} \left( 1 + y \cdot \frac{|1 + 2 \cos(4t\theta(p)) + 2 \cos(3t\theta(p))|}{p} \right)  \\
        &< \frac{1}{T} x (\log x)^{\frac{y}{2} - 1} + \sum_{t \leq T} \frac{x}{t \log x} \cdot \prod_{\substack{p \leq x \\ p \equiv 1 \text{ mod } 4}} \left( 1 + y \cdot \frac{|1 + 2 \cos(4t\theta(p)) + 2 \cos(3t\theta(p))|}{p} \right).
    \end{split}
    \end{align*}

    We now focus on understanding the product term from the equation above. In particular, we focus on understanding the following expression:
    \begin{equation} \label{eq:thm:equid_2}
        \sum_{p \in \mathbb{G}(u)} \frac{1}{p} |1 + 2 \cos(4t \theta(p)) + 2 \cos(3t \theta(p))|
    \end{equation}
    The period of a trigonometric function $y = 1 + 2 \cos (4t \theta) + 2 \cos(3t \theta)$ is equal to $\frac{\pi}{t}$, and in particular over the interval $E = [-\frac{\pi}{2t}, \frac{\pi}{2t}]$, the function is non-negative over the intervals $E_1 = [-\frac{\pi}{14t}, \frac{\pi}{14t}]$ and $E_3 = [-\frac{5\pi}{14t}, -\frac{3\pi}{14t}] \cup [\frac{3\pi}{14t}, \frac{5\pi}{14t}]$, and non-positive over the intervals $E_2 = [-\frac{3\pi}{14t}, -\frac{\pi}{14t}] \cup [\frac{\pi}{14t}, \frac{3\pi}{14t}]$ and $E_4 = [-\frac{\pi}{2t}, -\frac{5\pi}{14t}] \cup [\frac{5\pi}{14t}, \frac{\pi}{2t}]$. Equation (\ref{eq:thm:equid_2}) can hence be rewritten as
    \begin{equation}
        (\ref{eq:thm:equid_2}) = \frac{t}{4} \sum_{p \in \mathbb{G}(u)} \sum_{i=1}^4 (-1)^{i+1} \cdot \left( \sum_{\theta(p) \in E_i} \int_{\theta(p)}^{e_{i}^{sup}} 4t \sin(4t \phi) + 3t \sin(3t\phi) d\phi \right)
    \end{equation}
    where $e_i^{sup} := \text{sup}\{x \; | \; x \in E_i\}$. Note that the extra term $\frac{t}{4}$ originates from the fact that the period of the function $y = 1 + 2 \cos (4t \theta) + 2 \cos(3t \theta)$ is equal to $\frac{\pi}{t}$, and that $\theta(p)$ lies in the first octant of the xy-plane. By Fubini's theorem, we obtain
    \begin{equation}
        (\ref{eq:thm:equid_2}) = \frac{t}{4} \sum_{p \in \mathbb{G}(u)} \sum_{i=1}^4 (-1)^{i+1} \cdot \left(\int_{E_i} \left[ \sum_{p \in \mathbb{G}(u), e_i^{inf} < \theta(p) \leq \phi} 1 \right] \times (4t \sin(4t \phi) + 3t \sin(3t\phi) d\phi \right)
    \end{equation}
    where $e_i^{inf} := \text{inf}\{x \; | \; x \in E_i\}$. 

    To evaluate the integrals, we recall Kubilius's theorem \cite{Ku50}.
    \begin{theorem}[\cite{Ku50}]
        The number of Gaussian prime $w$ in the sector $0 \leq \alpha \leq \text{arg}(w) \leq \beta \leq 2 \pi$, and $|w|^2 \leq u$ is equal to:
        \begin{equation}
            \frac{2}{\pi}(\beta - \alpha) \int_2^u \frac{1}{\log v} dv + O(u \text{Exp}[-b \cdot \sqrt{\log u}])
        \end{equation}
        where $b$ is a positive absolute constant.
    \end{theorem}
    Kubilius's theorem allows us to rewrite the desired sum as
    \begin{align}
    \begin{split}
        (\ref{eq:thm:equid_2}) &= \frac{t}{4} \left[ \frac{2}{\pi} \int_2^u \frac{1}{\log v} dv \right] \cdot \left( \sum_{i=1}^4 (-1)^{i+1} \cdot \int_{E_i} (\phi - e_i^{inf})(4t \sin(4t \phi) + 3t \sin(3t \phi)) d \phi \right) + O(u \cdot e^{-b \sqrt{\log u}}) \\
        &= C \int_2^u \frac{dv}{\log v} + O(t \cdot u \cdot e^{-b \sqrt{\log u}}).
    \end{split}
    \end{align}
    where
    \begin{equation*}
        C = \frac{4 \sin(\pi/14) + 6\sin(\pi/7) - 8\sin(3\pi/14) + 3 \cos(\pi/14) + 8 \cos(\pi/7) - 6 \cos(3\pi/14)}{6\pi}.
    \end{equation*}
    By using Abel's partial summation formula, for any $2 \leq \omega \leq x$, we obtain
    \begin{align} \label{eqn:equid_3}
        \begin{split}
            & \; \; \; \; \; \; \; \; \sum_{p \in \mathbb{G}(x)} \frac{1}{p} |1 + 2 \cos(4t\theta(p)) + 2 \cos(3t \theta(p))| \\
            &\leq \frac{1}{2} \log \log w + C \log \left( \frac{\log x}{\log \omega} \right) + O(1) + O(t \cdot e^{-b \sqrt{\log w}})
        \end{split}
    \end{align}
    We may choose $\omega = (\frac{1}{b} \cdot \log t)^2$ to get a uniform bound for $t \in 4 \mathbb{N}$:
    \begin{equation}
        (\ref{eqn:equid_3}) \leq C \log \log x + (1-2C) \log \log t + O(1)
    \end{equation}
    Apply the above inequality to $\sum_{n \in \mathbb{G}(x)} \frac{\Delta_\mathfrak{d}(n)}{f_\mathfrak{d}(n)}y^{\Omega(n)}$ to achieve
    \begin{equation} \label{eq:equid_4}
        \sum_{n in \mathbb{G}(x)} \frac{\Delta_\mathfrak{d}(n)}{f_\mathfrak{d}(n)}y^{\Omega(n)} \ll \frac{1}{T} x (\log x)^{\frac{y}{2} - 1} + x (\log x)^{C \cdot y - 1} \cdot (\log T)^{(1 - 2C) \cdot y + 1}
    \end{equation}
    Choose $T = \log x$ and $y = \frac{1}{2C}$. Then one achieves that
    \begin{equation}
        \sum_{n \in \mathbb{G}(x)} \frac{\Delta_\mathfrak{d}(n)}{f_\mathfrak{d}(n)}y^{\Omega(n)} \ll (A + o(1)) \cdot \frac{x}{\sqrt{\log x}} (\log \log x)^{\frac{1}{2C}}
    \end{equation}
    where $A = \frac{1}{2\sqrt{2}} \cdot \prod_{q} \left( 1 - \frac{1}{q^2} \right)^{\frac{1}{2}}$. For some $\epsilon > 0$, for all but $o \left( \frac{x}{\sqrt{\log x}} \right)$ of the integers in $\mathbb{G}(x)$, we obtain that $\Omega(n) > (\frac{1}{2} - \epsilon) \cdot \log \log x$. This fact, along with (\ref{eq:equid_4}), proves the desired statement on angular equidistribution of Gaussian integers satisfying prime divisibility conditions, as outlined in (\ref{eq:equid_0}). Note that the rate of convergence $\kappa$ is obtained as $\kappa = -\frac{1}{2}\log(2C)$.
\end{proof}

\begin{definition}
    We denote by $H(\theta)$ the function over $[0,2\pi]$ given by
    \begin{equation}
        H(\theta) := \max\left[ \frac{1}{2} \cdot |123 \cos \theta + 114 \sin \theta + 125|^3, |2502 \cos \theta + 261 \sin \theta + 2500|^2 \right]
    \end{equation}
    Denote by $H_{max} := \max_{\theta \in [0,2\pi]} H(\theta)$ and $H_{min} := \min_{\theta \in [0,2\pi]} H(\theta)$.
\end{definition}
We have $H(a,b,c) = 2^6 \cdot 3^3 \cdot c^6 \cdot H(\theta)$ with respect to change of variables $a = c^2 \cos \theta$ and $b = c^2 \sin \theta$.

\begin{definition}
    We denote by $\varphi_{H}: \left[1,\sqrt[6]{\frac{H_{max}}{H_{min}}} \right] \to [0,1]$ a continuous decreasing piecewise differentiable function defined as
    \begin{equation}
        \varphi_H(x) := 1 - \frac{1}{2\pi} \mu \left( \left\{ \theta \in [0,2\pi] \; : \; x \geq \sqrt[6]{\frac{H_{max}}{H(\theta)}} \right\} \right)
    \end{equation}
\end{definition}

We now have all the ingredients to count $5$-isogenies of elliptic curves over $\mathbb{Q}$ up to twist.

\begin{theorem}
    There exists an explicitly computable constant $f_{1}^* > 0$ such that for every $\epsilon > 0$, we have
    \begin{equation*}
        N_5^{\tw}(T) = f_1^* T \log T + O(T(\log T)^{1-\kappa + \epsilon})
    \end{equation*}
    as $T \to \infty$, where $\kappa$ is the constant explicitly determined from Theorem \ref{theorem:equidistribution}. Furthermore, the constant $f_1^*$ is given by
    \begin{equation*}
        f_1^* = \frac{83\sqrt{3}}{24} \cdot f_1 \cdot \left(- \int_1^{\sqrt[6]{\frac{H_{max}}{H_{min}}}} z \varphi'_H(z) dz \right) \cdot \frac{1}{\sqrt[6]{H_{max}}}.
    \end{equation*}
\end{theorem}
\begin{proof}
    To be filled in.
\end{proof}


\clearpage
\bibliography{refs.bib}{} \bibliographystyle{amsalpha}
\end{document}
